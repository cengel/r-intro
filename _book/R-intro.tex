\documentclass[]{book}
\usepackage{lmodern}
\usepackage{amssymb,amsmath}
\usepackage{ifxetex,ifluatex}
\usepackage{fixltx2e} % provides \textsubscript
\ifnum 0\ifxetex 1\fi\ifluatex 1\fi=0 % if pdftex
  \usepackage[T1]{fontenc}
  \usepackage[utf8]{inputenc}
\else % if luatex or xelatex
  \ifxetex
    \usepackage{mathspec}
  \else
    \usepackage{fontspec}
  \fi
  \defaultfontfeatures{Ligatures=TeX,Scale=MatchLowercase}
\fi
% use upquote if available, for straight quotes in verbatim environments
\IfFileExists{upquote.sty}{\usepackage{upquote}}{}
% use microtype if available
\IfFileExists{microtype.sty}{%
\usepackage{microtype}
\UseMicrotypeSet[protrusion]{basicmath} % disable protrusion for tt fonts
}{}
\usepackage[margin=1in]{geometry}
\usepackage{hyperref}
\hypersetup{unicode=true,
            pdftitle={Introduction to R},
            pdfauthor={Claudia A Engel},
            pdfborder={0 0 0},
            breaklinks=true}
\urlstyle{same}  % don't use monospace font for urls
\usepackage{natbib}
\bibliographystyle{apalike}
\usepackage{color}
\usepackage{fancyvrb}
\newcommand{\VerbBar}{|}
\newcommand{\VERB}{\Verb[commandchars=\\\{\}]}
\DefineVerbatimEnvironment{Highlighting}{Verbatim}{commandchars=\\\{\}}
% Add ',fontsize=\small' for more characters per line
\usepackage{framed}
\definecolor{shadecolor}{RGB}{248,248,248}
\newenvironment{Shaded}{\begin{snugshade}}{\end{snugshade}}
\newcommand{\KeywordTok}[1]{\textcolor[rgb]{0.13,0.29,0.53}{\textbf{{#1}}}}
\newcommand{\DataTypeTok}[1]{\textcolor[rgb]{0.13,0.29,0.53}{{#1}}}
\newcommand{\DecValTok}[1]{\textcolor[rgb]{0.00,0.00,0.81}{{#1}}}
\newcommand{\BaseNTok}[1]{\textcolor[rgb]{0.00,0.00,0.81}{{#1}}}
\newcommand{\FloatTok}[1]{\textcolor[rgb]{0.00,0.00,0.81}{{#1}}}
\newcommand{\ConstantTok}[1]{\textcolor[rgb]{0.00,0.00,0.00}{{#1}}}
\newcommand{\CharTok}[1]{\textcolor[rgb]{0.31,0.60,0.02}{{#1}}}
\newcommand{\SpecialCharTok}[1]{\textcolor[rgb]{0.00,0.00,0.00}{{#1}}}
\newcommand{\StringTok}[1]{\textcolor[rgb]{0.31,0.60,0.02}{{#1}}}
\newcommand{\VerbatimStringTok}[1]{\textcolor[rgb]{0.31,0.60,0.02}{{#1}}}
\newcommand{\SpecialStringTok}[1]{\textcolor[rgb]{0.31,0.60,0.02}{{#1}}}
\newcommand{\ImportTok}[1]{{#1}}
\newcommand{\CommentTok}[1]{\textcolor[rgb]{0.56,0.35,0.01}{\textit{{#1}}}}
\newcommand{\DocumentationTok}[1]{\textcolor[rgb]{0.56,0.35,0.01}{\textbf{\textit{{#1}}}}}
\newcommand{\AnnotationTok}[1]{\textcolor[rgb]{0.56,0.35,0.01}{\textbf{\textit{{#1}}}}}
\newcommand{\CommentVarTok}[1]{\textcolor[rgb]{0.56,0.35,0.01}{\textbf{\textit{{#1}}}}}
\newcommand{\OtherTok}[1]{\textcolor[rgb]{0.56,0.35,0.01}{{#1}}}
\newcommand{\FunctionTok}[1]{\textcolor[rgb]{0.00,0.00,0.00}{{#1}}}
\newcommand{\VariableTok}[1]{\textcolor[rgb]{0.00,0.00,0.00}{{#1}}}
\newcommand{\ControlFlowTok}[1]{\textcolor[rgb]{0.13,0.29,0.53}{\textbf{{#1}}}}
\newcommand{\OperatorTok}[1]{\textcolor[rgb]{0.81,0.36,0.00}{\textbf{{#1}}}}
\newcommand{\BuiltInTok}[1]{{#1}}
\newcommand{\ExtensionTok}[1]{{#1}}
\newcommand{\PreprocessorTok}[1]{\textcolor[rgb]{0.56,0.35,0.01}{\textit{{#1}}}}
\newcommand{\AttributeTok}[1]{\textcolor[rgb]{0.77,0.63,0.00}{{#1}}}
\newcommand{\RegionMarkerTok}[1]{{#1}}
\newcommand{\InformationTok}[1]{\textcolor[rgb]{0.56,0.35,0.01}{\textbf{\textit{{#1}}}}}
\newcommand{\WarningTok}[1]{\textcolor[rgb]{0.56,0.35,0.01}{\textbf{\textit{{#1}}}}}
\newcommand{\AlertTok}[1]{\textcolor[rgb]{0.94,0.16,0.16}{{#1}}}
\newcommand{\ErrorTok}[1]{\textcolor[rgb]{0.64,0.00,0.00}{\textbf{{#1}}}}
\newcommand{\NormalTok}[1]{{#1}}
\usepackage{longtable,booktabs}
\usepackage{graphicx,grffile}
\makeatletter
\def\maxwidth{\ifdim\Gin@nat@width>\linewidth\linewidth\else\Gin@nat@width\fi}
\def\maxheight{\ifdim\Gin@nat@height>\textheight\textheight\else\Gin@nat@height\fi}
\makeatother
% Scale images if necessary, so that they will not overflow the page
% margins by default, and it is still possible to overwrite the defaults
% using explicit options in \includegraphics[width, height, ...]{}
\setkeys{Gin}{width=\maxwidth,height=\maxheight,keepaspectratio}
\IfFileExists{parskip.sty}{%
\usepackage{parskip}
}{% else
\setlength{\parindent}{0pt}
\setlength{\parskip}{6pt plus 2pt minus 1pt}
}
\setlength{\emergencystretch}{3em}  % prevent overfull lines
\providecommand{\tightlist}{%
  \setlength{\itemsep}{0pt}\setlength{\parskip}{0pt}}
\setcounter{secnumdepth}{5}
% Redefines (sub)paragraphs to behave more like sections
\ifx\paragraph\undefined\else
\let\oldparagraph\paragraph
\renewcommand{\paragraph}[1]{\oldparagraph{#1}\mbox{}}
\fi
\ifx\subparagraph\undefined\else
\let\oldsubparagraph\subparagraph
\renewcommand{\subparagraph}[1]{\oldsubparagraph{#1}\mbox{}}
\fi

%%% Use protect on footnotes to avoid problems with footnotes in titles
\let\rmarkdownfootnote\footnote%
\def\footnote{\protect\rmarkdownfootnote}

%%% Change title format to be more compact
\usepackage{titling}

% Create subtitle command for use in maketitle
\newcommand{\subtitle}[1]{
  \posttitle{
    \begin{center}\large#1\end{center}
    }
}

\setlength{\droptitle}{-2em}
  \title{Introduction to R}
  \pretitle{\vspace{\droptitle}\centering\huge}
  \posttitle{\par}
  \author{Claudia A Engel}
  \preauthor{\centering\large\emph}
  \postauthor{\par}
  \predate{\centering\large\emph}
  \postdate{\par}
  \date{Last updated: August 24, 2017}

\usepackage{booktabs}
\usepackage{amsthm}
\makeatletter
\def\thm@space@setup{%
  \thm@preskip=8pt plus 2pt minus 4pt
  \thm@postskip=\thm@preskip
}
\makeatother

\usepackage{amsthm}
\newtheorem{theorem}{Theorem}[chapter]
\newtheorem{lemma}{Lemma}[chapter]
\theoremstyle{definition}
\newtheorem{definition}{Definition}[chapter]
\newtheorem{corollary}{Corollary}[chapter]
\newtheorem{proposition}{Proposition}[chapter]
\theoremstyle{definition}
\newtheorem{example}{Example}[chapter]
\theoremstyle{remark}
\newtheorem*{remark}{Remark}
\begin{document}
\maketitle

{
\setcounter{tocdepth}{1}
\tableofcontents
}
\chapter*{Prerequisites}\label{prerequisites}
\addcontentsline{toc}{chapter}{Prerequisites}

\begin{itemize}
\tightlist
\item
  Geared specifically towards users who are \textbf{new} to R.\\
\item
  Have \href{https://cran.r-project.org/}{R} and
  \href{https://www.rstudio.com/}{RStudio} installed (see setup
  instructions below).
\end{itemize}

\textbf{Next workshop:\\
Thu, October 19, 2017, 1:30-4pm,\\
Room 433A in CESTA (Building 160 - Wallenberg 4th floor)}

\section*{Setup Instructions}\label{setup-instructions}
\addcontentsline{toc}{section}{Setup Instructions}

\href{https://cran.r-project.org/}{\textbf{R}} and
\href{https://www.rstudio.com/}{\textbf{RStudio}} are separate downloads
and installations. R is the underlying statistical computing
environment, but using R alone is no fun. RStudio is a graphical
integrated development environment (IDE) that makes using R much easier
and more interactive. You need to install R before you install RStudio.

\subsection*{macOS}\label{macos}
\addcontentsline{toc}{subsection}{macOS}

\subsubsection*{If you already have R and RStudio
installed}\label{if-you-already-have-r-and-rstudio-installed}
\addcontentsline{toc}{subsubsection}{If you already have R and RStudio
installed}

\begin{itemize}
\tightlist
\item
  Open RStudio, and click on ``Help'' \textgreater{} ``Check for
  updates''. If a new version is available, quit RStudio, and download
  the latest version for RStudio.
\item
  To check the version of R you are using, start RStudio and the first
  thing that appears on the terminal indicates the version of R you are
  running. Alternatively, you can type \texttt{sessionInfo()}, which
  will also display which version of R you are running. Go on the
  \href{https://cran.r-project.org/bin/macosx/}{CRAN website} and check
  whether a more recent version is available. If so, please download and
  install it.
\end{itemize}

\subsubsection*{If you don't have R and RStudio
installed}\label{if-you-dont-have-r-and-rstudio-installed}
\addcontentsline{toc}{subsubsection}{If you don't have R and RStudio
installed}

\begin{itemize}
\tightlist
\item
  Download R from the \href{http://cran.r-project.org/bin/macosx}{CRAN
  website}.
\item
  Select the \texttt{.pkg} file for the latest R version
\item
  Double click on the downloaded file to install R
\item
  Go to the
  \href{https://www.rstudio.com/products/rstudio/download/\#download}{RStudio
  download page}
\item
  Under \emph{Installers} select \textbf{RStudio x.yy.zzz - Mac OS X
  10.6+ (64-bit)} (where x, y, and z represent version numbers)
\item
  Double click the file to install RStudio
\item
  Once it's installed, open RStudio to make sure it works and you don't
  get any error messages.
\end{itemize}

\subsection*{Windows}\label{windows}
\addcontentsline{toc}{subsection}{Windows}

\subsubsection*{If you already have R and RStudio
installed}\label{if-you-already-have-r-and-rstudio-installed-1}
\addcontentsline{toc}{subsubsection}{If you already have R and RStudio
installed}

\begin{itemize}
\tightlist
\item
  Open RStudio, and click on ``Help'' \textgreater{} ``Check for
  updates''. If a new version is available, quit RStudio, and download
  the latest version for RStudio.
\item
  To check which version of R you are using, start RStudio and the first
  thing that appears in the console indicates the version of R you are
  running. Alternatively, you can type \texttt{sessionInfo()}, which
  will also display which version of R you are running. Go on the
  \href{https://cran.r-project.org/bin/windows/base/}{CRAN website} and
  check whether a more recent version is available. If so, please
  download and install it. You can
  \href{https://cran.r-project.org/bin/windows/base/rw-FAQ.html\#How-do-I-UNinstall-R_003f}{check
  here} for more information on how to remove old versions from your
  system if you wish to do so.
\end{itemize}

\subsubsection*{If you don't have R and RStudio
installed}\label{if-you-dont-have-r-and-rstudio-installed-1}
\addcontentsline{toc}{subsubsection}{If you don't have R and RStudio
installed}

\begin{itemize}
\tightlist
\item
  Download R from the
  \href{http://cran.r-project.org/bin/windows/base/release.htm}{CRAN
  website}.
\item
  Run the \texttt{.exe} file that was just downloaded
\item
  Go to the
  \href{https://www.rstudio.com/products/rstudio/download/\#download}{RStudio
  download page}
\item
  Under \emph{Installers} select \textbf{RStudio x.yy.zzz - Windows
  XP/Vista/7/8} (where x, y, and z represent version numbers)
\item
  Double click the file to install it
\item
  Once it's installed, open RStudio to make sure it works and you don't
  get any error messages.
\end{itemize}

\subsection*{Linux}\label{linux}
\addcontentsline{toc}{subsection}{Linux}

\begin{itemize}
\tightlist
\item
  Follow the instructions for your distribution from
  \href{https://cloud.r-project.org/bin/linux}{CRAN}, they provide
  information to get the most recent version of R for common
  distributions. For most distributions, you could use your package
  manager (e.g., for Debian/Ubuntu run
  \texttt{sudo\ apt-get\ install\ r-base}, and for Fedora
  \texttt{sudo\ yum\ install\ R}), but we don't recommend this approach
  as the versions provided by this are usually out of date. In any case,
  make sure you have at least R 3.3.1.
\item
  Go to the
  \href{https://www.rstudio.com/products/rstudio/download/\#download}{RStudio
  download page}
\item
  Under \emph{Installers} select the version that matches your
  distribution, and install it with your preferred method (e.g., with
  Debian/Ubuntu \texttt{sudo\ dpkg\ -i\ \ \ rstudio-x.yy.zzz-amd64.deb}
  at the terminal).
\item
  Once it's installed, open RStudio to make sure it works and you don't
  get any error messages.
\end{itemize}

\section*{Acknowledgements}\label{acknowledgements}
\addcontentsline{toc}{section}{Acknowledgements}

Part of the materials for this tutorial are adapted from
\url{http://datacarpentry.org} and \url{http://softwarecarpentry.org}.

\chapter{R and Rstudio}\label{backgroud}

\begin{quote}
Learning Objectives

\begin{itemize}
\tightlist
\item
  Be familiar with reasons to use R.
\item
  Understand how R relates to RStudio.
\item
  Be able to navigate the RStudio interface including the Script,
  Console, Environment, Help, Files, and Plots windows.
\item
  Create an R Project in RStudio.
\item
  Set a ``working'' directory.
\item
  Send commands from the Script window to the Console in RStudio.
\end{itemize}
\end{quote}

\begin{center}\rule{0.5\linewidth}{\linethickness}\end{center}

\section{What is R? What is RStudio?}\label{what-is-r-what-is-rstudio}

The term ``R'' is used to refer to both the programming language to
write scripts and the software (``environment'') that interprets the
scripts written in R. It is an alternative to statistical packages like
SAS, SPSS, or Stata, which lets you perform a wide variety of data
analysis, statistics, and visualization.

RStudio is currently a very popular way to not only write your R scripts
but also to interact with the R software. To function correctly, RStudio
needs R and therefore both need to be installed on your computer.

\section{Why learn R?}\label{why-learn-r}

\subsection{R does not involve lots of pointing and clicking, and that's
a good
thing}\label{r-does-not-involve-lots-of-pointing-and-clicking-and-thats-a-good-thing}

The learning curve might be steeper than with other software, but with
R, the results of your analysis does not rely on remembering a
succession of pointing and clicking, but instead on a series of written
commands, and that's a good thing! So, if you want to redo your analysis
because you collected more data, you don't have to remember which button
you clicked in which order to obtain your results, you just have to run
your script again.

Working with scripts makes the steps you used in your analysis clear,
and the code you write can be inspected by someone else who can give you
feedback and spot mistakes.

Working with scripts forces you to have a deeper understanding of what
you are doing, and facilitates your learning and comprehension of the
methods you use.

\subsection{R code is great for
reproducibility}\label{r-code-is-great-for-reproducibility}

Reproducibility is when someone else (including your future self) can
obtain the same results from the same dataset when using the same
analysis.

R integrates with other tools to generate manuscripts from your code. If
you collect more data, or fix a mistake in your dataset, the figures and
the statistical tests in your manuscript are updated automatically.

An increasing number of journals and funding agencies expect analyses to
be reproducible, so knowing R will give you an edge with these
requirements.

\subsection{R is interdisciplinary and
extensible}\label{r-is-interdisciplinary-and-extensible}

With 10,000+ packages that can be installed to extend its capabilities,
R provides a framework that allows you to combine statistical approaches
from many scientific disciplines to best suit the analytical framework
you need to analyze your data. For instance, R has packages for image
analysis, mapping, time series, text mining, and a lot more.

\subsection{R works on data of all shapes and
sizes}\label{r-works-on-data-of-all-shapes-and-sizes}

The skills you learn with R scale easily with the size of your dataset.
Whether your dataset has hundreds or millions of lines, it won't make
much difference to you.

R is designed for data analysis. It comes with special data structures
and data types that make handling of missing data and statistical
factors convenient.

R can connect to spreadsheets, databases, and many other data formats,
on your computer or on the web.

\subsection{R produces high-quality
graphics}\label{r-produces-high-quality-graphics}

The plotting functionalities in R are endless, and allow you to adjust
any aspect of your graph to convey most effectively the message from
your data.

\subsection{R has a large community}\label{r-has-a-large-community}

Thousands of people use R daily. Many of them are willing to help you
through mailing lists and websites such as
\href{https://stackoverflow.com/questions/tagged/r}{Stack Overflow}.

\subsection{Not only is R free, but it is also open-source and
cross-platform}\label{not-only-is-r-free-but-it-is-also-open-source-and-cross-platform}

Anyone can inspect the source code to see how R works. Because of this
transparency, there is less chance for mistakes, and if you (or someone
else) find some, you can report and fix bugs.

\section{Knowing your way around
RStudio}\label{knowing-your-way-around-rstudio}

Let's start by learning about \href{https://www.rstudio.com/}{RStudio},
which is an Integrated Development Environment (IDE) for working with R.

The RStudio IDE open-source product is free under the
\href{https://www.gnu.org/licenses/agpl-3.0.en.html}{Affero General
Public License (AGPL) v3}. The RStudio IDE is also available with a
commercial license and priority email support from RStudio, Inc.

We will use RStudio IDE to write code, navigate the files on our
computer, inspect the variables we are going to create, and visualize
the plots we will generate. RStudio can also be used for other things
(e.g., version control, developing packages, writing Shiny apps) that we
will not cover during the workshop.

\begin{figure}
\includegraphics[width=1\linewidth]{img/rstudio-screenshot} \caption{The RStudio Interface}\label{fig:RStudio-GUI}
\end{figure}

RStudio is divided into 4 ``Panes'': the \textbf{Source} for your
scripts and documents (top-left, in the default layout), the R
\textbf{Console} (bottom-left), your \textbf{Environment/History}
(top-right), and your \textbf{Files/Plots/Packages/Help/Viewer}
(bottom-right). The placement of these panes and their content can be
customized (see menu, Tools -\textgreater{} Global Options
-\textgreater{} Pane Layout). One of the advantages of using RStudio is
that all the information you need to write code is available in a single
window. Additionally, with many shortcuts, autocompletion, and
highlighting for the major file types you use while developing in R,
RStudio will make typing easier and less error-prone.

\section{How to start an R project}\label{how-to-start-an-r-project}

It is good practice to keep a set of related data, analyses, and text
self-contained in a single folder. When working with R and RStudio you
typically want that single top folder to be the folder you are working
in. In order to tell R this, you will want to set that folder as your
\textbf{working directory}. Whenever you refer to other scripts or data
or directories contained within the working directory you can then use
\emph{relative paths} to files that indicate where inside the project a
file is located. (That is opposed to absolute paths, which point to
where a file is on a specific computer). having everything contained in
a single directory makes it a lot easier to move your project around on
your computer and share it with others without worrying about whether or
not the underlying scripts will still work.

Whenever you create a project with RStudio it creates a working
directory for you and remembers its location (allowing you to quickly
navigate to it) and optionally preserves custom settings and open files
to make it easier to resume work after a break. Below, we will go
through the steps for creating an ``R Project'' for this workshop.

\begin{itemize}
\tightlist
\item
  Start RStudio
\item
  Under the \texttt{File} menu, click on \texttt{New\ project}, choose
  \texttt{New\ directory}, then \texttt{Empty\ project}
\item
  As directory (or folder) name enter \texttt{r-intro} and create
  project as subdirecory of your desktop folder:
  \texttt{\textasciitilde{}/Desktop}\\
\item
  Click on \texttt{Create\ project}
\item
  Under the \texttt{Files} tab on the right of the screen, click on
  \texttt{New\ Folder} and create a folder named \texttt{data} within
  your newly created working directory (e.g.,
  \texttt{\textasciitilde{}/r-intro/data})
\item
  Download the \href{./code-handout.R}{code handout}, place it in your
  working directory and rename it (e.g., \texttt{r-intro-script.R}).
\end{itemize}

Your working directory should now look like this:

\begin{figure}
\includegraphics[width=0.8\linewidth]{img/Rproject-setup} \caption{What it should look like at the beginning of this lesson}\label{fig:working-dir}
\end{figure}

If you ever need to set a different working directory you can use the
RStudio interface like this:

\begin{figure}
\includegraphics[width=0.6\linewidth]{img/setWD} \caption{How to set a working directory with the RStudio interface}\label{fig:set-working-dir}
\end{figure}

Alternatively, you can use the shortcut \texttt{Ctrl} + \texttt{Shift} +
\texttt{H} to set a working directory in RStudio.

To set a working directory in R go to the Console and type:

\begin{Shaded}
\begin{Highlighting}[]
\KeywordTok{setwd}\NormalTok{(}\StringTok{"Path/To/Your/Workingdirectory"}\NormalTok{)}
\end{Highlighting}
\end{Shaded}

If you need to check which working directory R thinks it is in:

\begin{Shaded}
\begin{Highlighting}[]
\KeywordTok{getwd}\NormalTok{()}
\end{Highlighting}
\end{Shaded}

\subsection{Organizing your working
directory}\label{organizing-your-working-directory}

Using a consistent folder structure across your projects will help keep
things organized, and will also make it easy to find/file things in the
future. This can be especially helpful when you have multiple projects.
In general, you may create directories (folders) for \textbf{scripts},
\textbf{data}, and \textbf{documents}.

\begin{itemize}
\tightlist
\item
  \textbf{\texttt{data/}} Use this folder to store your raw data and
  intermediate datasets you may create for the need of a particular
  analysis. For the sake of transparency and
  \href{https://en.wikipedia.org/wiki/Provenance}{provenance}, you
  should \emph{always} keep a copy of your raw data accessible and do as
  much of your data cleanup and preprocessing programmatically (i.e.,
  with scripts, rather than manually) as possible. Separating raw data
  from processed data is also a good idea. For example, you could have
  subfolders in your \texttt{data} directory named \texttt{data/raw/}
  and \texttt{data/processed} that woudl contain the respective raw and
  processed files. I also like to log my data processing steps in a
  simple textfile that I keep there as well.
\item
  \textbf{\texttt{documents/}} If you are wroking on a paper this would
  be a place to keep outlines, drafts, and other text.
\item
  \textbf{\texttt{scripts/}} This would be the location to keep your R
  scripts. Again, depending on the complexity, you may want to add
  subfolders that contain, for example all the plotting scripts, or all
  the datas cleaning scripts.
\end{itemize}

You may want additional directories or subdirectories depending on your
project needs, but this is a good template to form the backbone of your
working directory.

\section{Interacting with R}\label{interacting-with-r}

The basis of programming is that we write down instructions for the
computer to follow, and then we tell the computer to follow those
instructions. We write, or \emph{code}, instructions in R because it is
a common language that both the computer and we can understand. We call
the instructions \emph{commands} and we tell the computer to follow the
instructions by \emph{executing} (also called \emph{running}) those
commands.

There are two main ways of interacting with R: by using the
\textbf{console} or by using \textbf{script files} (plain text files
that contain your code).

\subsection{RStudio Console}\label{rstudio-console}

The console pane in RStudio is the place where commands written in the R
language can be typed and executed immediately by the computer. It is
also where the results will be shown for commands that have been
executed. You can type commands directly into the console and press
\texttt{Enter} to execute those commands, but they will be forgotten
when you close the session.

\begin{quote}
Challenge

\begin{itemize}
\tightlist
\item
  Use R to determine what your working directory is.
\item
  Use R to change your working directory to some other place. What do
  you notice in the RStudio Files window?
\item
  Use RStudio to change back to your previous working directory
  (r-intro) What do you notice in the RStudio Console?
\end{itemize}
\end{quote}

\subsection{RStudio Script Editor}\label{rstudio-script-editor}

Because we want our code and workflow to be reproducible, it is better
to type the commands we want in the script editor, and save the script.
This way, there is a complete record of what we did, and anyone
(including our future selves!) can easily replicate the results on their
computer.

RStudio allows you to execute commands directly from the script editor
by using the \texttt{Ctrl} + \texttt{Enter} shortcut (on Macs,
\texttt{Cmd} + \texttt{Return} will work, too). The command on the
current line in the script (indicated by the cursor) or all of the
commands in the currently selected text will be sent to the console and
executed when you press \texttt{Ctrl} + \texttt{Enter}. You can find
other keyboard shortcuts in this
\href{https://github.com/rstudio/cheatsheets/blob/master/source/pdfs/rstudio-IDE-cheatsheet.pdf}{RStudio
cheatsheet about the RStudio IDE}.

At some point in your analysis you may want to check the content of a
variable or the structure of an object, without necessarily keeping a
record of it in your script. You can type these commands and execute
them directly in the console. RStudio provides the \texttt{Ctrl} +
\texttt{1} and \texttt{Ctrl} + \texttt{2} shortcuts allow you to jump
between the script and the console panes.

\subsection{The R command prompt}\label{the-r-command-prompt}

If R is ready to accept commands, the R console by default shows a
\texttt{\textgreater{}} prompt. If it receives a command (by typing,
copy-pasting or sent from the script editor using \texttt{Ctrl} +
\texttt{Enter}), R will try to execute it, and when ready, will show the
results and come back with a new \texttt{\textgreater{}} prompt to wait
for new commands.

If R is still waiting for you to enter more data because it isn't
complete yet, the console will show a \texttt{+} prompt. It means that
you haven't finished entering a complete command. This is because you
have not `closed' a parenthesis or quotation, i.e.~you don't have the
same number of left-parentheses as right-parentheses, or the same number
of opening and closing quotation marks. When this happens, and you
thought you finished typing your command, click inside the console
window and press \texttt{Esc}; this will cancel the incomplete command
and return you to the \texttt{\textgreater{}} prompt.

\chapter{Getting Started with R}\label{gettingstarted}

\begin{quote}
Learning Objectives

\begin{itemize}
\tightlist
\item
  Create R objects and and assign values to them.
\item
  Use comments to inform script.
\item
  Do simple arithmetic operations in R using values and objects.
\item
  Call functions with arguments and change their default options.
\item
  Inspect the content of vectors and manipulate their content.
\item
  Subset and extract values from vectors.
\item
  Correctly define and handle missing values in vectors.
\item
  Use the built-in RStudio help interface
\item
  Interpret the R help documentation
\item
  Provide sufficient information for troubleshooting with the R user
  community.
\item
  Download, install, and load R packages.
\end{itemize}
\end{quote}

\begin{center}\rule{0.5\linewidth}{\linethickness}\end{center}

\section{Creating objects in R}\label{creating-objects-in-r}

To do useful and interesting things in R, we need to assign
\emph{values} to \emph{objects}. To create an object, we need to give it
a name followed by the assignment operator \texttt{\textless{}-}, and
the value we want to give it:

\begin{Shaded}
\begin{Highlighting}[]
\NormalTok{weight_kg <-}\StringTok{ }\DecValTok{55}
\end{Highlighting}
\end{Shaded}

\texttt{\textless{}-} is the assignment operator. It assigns values on
the right to objects on the left. So, after executing
\texttt{x\ \textless{}-\ 3}, the value of \texttt{x} is \texttt{3}. The
arrow can be read as 3 \textbf{goes into} \texttt{x}. For historical
reasons, you can also use \texttt{=} for assignments, but not in every
context. Because of the
\href{http://blog.revolutionanalytics.com/2008/12/use-equals-or-arrow-for-assignment.html}{slight}
\href{https://web.archive.org/web/20130610005305/https://stat.ethz.ch/pipermail/r-help/2009-March/191462.html}{differences}
in syntax, it is good practice to always use \texttt{\textless{}-} for
assignments.

In RStudio, typing Alt + - (push Alt at the same time as the - key) will
write \texttt{\textless{}-} in a single keystroke.

Objects can be given any name such as \texttt{x},
\texttt{current\_temperature}, or \texttt{subject\_id}. You want your
object names to be explicit and not too long. They cannot start with a
number (\texttt{2x} is not valid, but \texttt{x2} is). R is case
sensitive (e.g., \texttt{weight\_kg} is different from
\texttt{Weight\_kg}). There are some names that cannot be used because
they are the names of fundamental functions in R (e.g., \texttt{if},
\texttt{else}, \texttt{for}, see
\href{https://stat.ethz.ch/R-manual/R-devel/library/base/html/Reserved.html}{here}
for a complete list). In general, even if it's allowed, it's best to not
use other function names (e.g., \texttt{c}, \texttt{T}, \texttt{mean},
\texttt{data}, \texttt{df}, \texttt{weights}). If in doubt, check the
help to see if the name is already in use. It's also best to avoid dots
(\texttt{.}) within a variable name as in \texttt{my.dataset}. There are
many functions in R with dots in their names for historical reasons, but
because dots have a special meaning in R (for methods) and other
programming languages, it's best to avoid them. It is also recommended
to use nouns for variable names, and verbs for function names. It's
important to be consistent in the styling of your code (where you put
spaces, how you name variables, etc.). Using a consistent coding style
makes your code clearer to read for your future self and your
collaborators. In R, three popular style guides are
\href{https://google.github.io/styleguide/Rguide.xml}{Google's},
\href{http://jef.works/R-style-guide/}{Jean Fan's} and the
\href{http://style.tidyverse.org/}{tidyverse's}. The tidyverse's is very
comprehensive and may seem overwhelming at first. You can install the
\href{https://github.com/jimhester/lintr}{\textbf{\texttt{lintr}}} to
automatically check for issues in the styling of your code.

When assigning a value to an object, R does not print anything. You can
force R to print the value by using parentheses or by typing the object
name:

\begin{Shaded}
\begin{Highlighting}[]
\NormalTok{weight_kg <-}\StringTok{ }\DecValTok{55}    \CommentTok{# doesn't print anything}
\NormalTok{(weight_kg <-}\StringTok{ }\DecValTok{55}\NormalTok{)  }\CommentTok{# but putting parenthesis around the call prints the value of `weight_kg`}
\NormalTok{weight_kg          }\CommentTok{# and so does typing the name of the object}
\end{Highlighting}
\end{Shaded}

Now that R has \texttt{weight\_kg} in memory, we can do arithmetic with
it. For instance, we may want to convert this weight into pounds (weight
in pounds is 2.2 times the weight in kg):

\begin{Shaded}
\begin{Highlighting}[]
\FloatTok{2.2} \NormalTok{*}\StringTok{ }\NormalTok{weight_kg}
\end{Highlighting}
\end{Shaded}

We can also change a variable's value by assigning it a new one:

\begin{Shaded}
\begin{Highlighting}[]
\NormalTok{weight_kg <-}\StringTok{ }\FloatTok{57.5}
\FloatTok{2.2} \NormalTok{*}\StringTok{ }\NormalTok{weight_kg}
\end{Highlighting}
\end{Shaded}

This means that assigning a value to one variable does not change the
values of other variables. For example, let's store the animal's weight
in pounds in a new variable, \texttt{weight\_lb}:

\begin{Shaded}
\begin{Highlighting}[]
\NormalTok{weight_lb <-}\StringTok{ }\FloatTok{2.2} \NormalTok{*}\StringTok{ }\NormalTok{weight_kg}
\end{Highlighting}
\end{Shaded}

and then change \texttt{weight\_kg} to 100.

\begin{Shaded}
\begin{Highlighting}[]
\NormalTok{weight_kg <-}\StringTok{ }\DecValTok{100}
\end{Highlighting}
\end{Shaded}

What do you think is the current content of the object
\texttt{weight\_lb}? 126.5 or 220?

\subsection{Comments}\label{comments}

The comment character in R is \texttt{\#}, anything to the right of a
\texttt{\#} in a script will be ignored by R. It is useful to leave
notes, and explanations in your scripts. RStudio makes it easy to
comment or uncomment a paragraph: after selecting the lines you want to
comment, press at the same time on your keyboard Ctrl + Shift + C. If
you only want to comment out one line, you can put the cursor at any
location of that line (i.e.~no need to select the whole line), then
press Ctrl + Shift + C.

\begin{quote}
Challenge

What are the values after each statement in the following?

\begin{Shaded}
\begin{Highlighting}[]
\NormalTok{mass <-}\StringTok{ }\FloatTok{47.5}            \CommentTok{# mass?}
\NormalTok{age  <-}\StringTok{ }\DecValTok{122}             \CommentTok{# age?}
\NormalTok{mass <-}\StringTok{ }\NormalTok{mass *}\StringTok{ }\FloatTok{2.0}      \CommentTok{# mass?}
\NormalTok{age  <-}\StringTok{ }\NormalTok{age -}\StringTok{ }\DecValTok{20}        \CommentTok{# age?}
\NormalTok{mass_index <-}\StringTok{ }\NormalTok{mass/age  }\CommentTok{# mass_index?}
\end{Highlighting}
\end{Shaded}
\end{quote}

\subsection{Functions and their
arguments}\label{functions-and-their-arguments}

Functions are ``canned scripts'' that automate more complicated sets of
commands including operations assignments, etc.

They all have in common that they are executed by typing their name
followed by round brackets, in which we provide one or more parameters
(or arguments) for the function to do something, separated by commas.
Each function requires their specific arguments and those can be looked
up with the help and all the arguments have names.

Many functions are predefined, or can be made available by importing R
\emph{packages} (more on that later). A function usually gets one or
more inputs called \emph{arguments}. Functions often (but not always)
return a \emph{value}. A typical example would be the function
\texttt{sqrt()}. The input (the argument) must be a number, and the
return value (in fact, the output) is the square root of that number.
Executing a function (`running it') is called \emph{calling} the
function. An example of a function call is:

\begin{Shaded}
\begin{Highlighting}[]
\NormalTok{b <-}\StringTok{ }\KeywordTok{sqrt}\NormalTok{(a)}
\end{Highlighting}
\end{Shaded}

Here, the value of \texttt{a} is given to the \texttt{sqrt()} function,
the \texttt{sqrt()} function calculates the square root, and returns the
value which is then assigned to variable \texttt{b}. This function is
very simple, because it takes just one argument.

The return `value' of a function need not be numerical (like that of
\texttt{sqrt()}), and it also does not need to be a single item: it can
be a set of things, or even a dataset. We'll see that when we read data
files into R.

Arguments can be anything, not only numbers or filenames, but also other
objects. Exactly what each argument means differs per function, and must
be looked up in the documentation (see below). Some functions take
arguments which may either be specified by the user, or, if left out,
take on a \emph{default} value: these are called \emph{options}. Options
are typically used to alter the way the function operates, such as
whether it ignores `bad values', or what symbol to use in a plot.
However, if you want something specific, you can specify a value of your
choice which will be used instead of the default.

Let's try a function that can take multiple arguments: \texttt{round()}.

\begin{Shaded}
\begin{Highlighting}[]
\KeywordTok{round}\NormalTok{(}\FloatTok{3.14159}\NormalTok{)}
\end{Highlighting}
\end{Shaded}

\begin{verbatim}
#> [1] 3
\end{verbatim}

Here, we've called \texttt{round()} with just one argument,
\texttt{3.14159}, and it has returned the value \texttt{3}. That's
because the default is to round to the nearest whole number. If we want
more digits we can see how to do that by getting information about the
\texttt{round} function. We can use \texttt{args(round)} or look at the
help for this function using \texttt{?round}.

\begin{Shaded}
\begin{Highlighting}[]
\KeywordTok{args}\NormalTok{(round)}
\end{Highlighting}
\end{Shaded}

\begin{verbatim}
#> function (x, digits = 0) 
#> NULL
\end{verbatim}

\begin{Shaded}
\begin{Highlighting}[]
\NormalTok{?round}
\end{Highlighting}
\end{Shaded}

We see that if we want a different number of digits, we can type
\texttt{digits=2} or however many we want.

\begin{Shaded}
\begin{Highlighting}[]
\KeywordTok{round}\NormalTok{(}\FloatTok{3.14159}\NormalTok{, }\DataTypeTok{digits =} \DecValTok{2}\NormalTok{)}
\end{Highlighting}
\end{Shaded}

\begin{verbatim}
#> [1] 3.14
\end{verbatim}

If you provide the arguments in the exact same order as they are defined
you don't have to name them:

\begin{Shaded}
\begin{Highlighting}[]
\KeywordTok{round}\NormalTok{(}\FloatTok{3.14159}\NormalTok{, }\DecValTok{2}\NormalTok{)}
\end{Highlighting}
\end{Shaded}

\begin{verbatim}
#> [1] 3.14
\end{verbatim}

And if you do name the arguments, you can switch their order:

\begin{Shaded}
\begin{Highlighting}[]
\KeywordTok{round}\NormalTok{(}\DataTypeTok{digits =} \DecValTok{2}\NormalTok{, }\DataTypeTok{x =} \FloatTok{3.14159}\NormalTok{)}
\end{Highlighting}
\end{Shaded}

\begin{verbatim}
#> [1] 3.14
\end{verbatim}

Note:

\begin{itemize}
\item
  R evaluates function arguments in three steps: first, by \emph{exact
  matching} on argument name, then by \emph{partial matching} on
  argument name, and finally by \emph{position}.
\item
  you \emph{do not have to} specify all of the arguments. If you don't,
  R will use default values if they are specified by the function. If no
  default value is specified, you will receive an error.
\end{itemize}

It's good practice to put the non-optional arguments (like the number
you're rounding) first in your function call, and to specify the names
of all optional arguments. If you don't, someone reading your code might
have to look up the definition of a function with unfamiliar arguments
to understand what you're doing.

Functions usually return someting back to you as output. Whatever they
return (a table, some informational text, a logical value, \ldots{}) is
by default written to the console, so you can see it right away.

Oftentimes, however, we want re-use the output of such a function. That
is when you assign the output to an R object to be accessed later on.

\subsection{Objects vs.~variables}\label{objects-vs.variables}

What are known as \texttt{objects} in \texttt{R} are known as
\texttt{variables} in many other programming languages. Depending on the
context, \texttt{object} and \texttt{variable} can have drastically
different meanings. However, in this lesson, the two words are used
synonymously. For more information see:
\url{https://cran.r-project.org/doc/manuals/r-release/R-lang.html\#Objects}

\section{Vectors and data types}\label{vectors-and-data-types}

A vector is the most common and basic data type in R, and is pretty much
the workhorse of R. A vector is composed by a series of values, which
can be either numbers or characters. We can assign a series of values to
a vector using the \texttt{c()} function. For example we can create a
vector of animal weights and assign it to a new object
\texttt{weight\_g}:

\begin{Shaded}
\begin{Highlighting}[]
\NormalTok{weight_g <-}\StringTok{ }\KeywordTok{c}\NormalTok{(}\DecValTok{50}\NormalTok{, }\DecValTok{60}\NormalTok{, }\DecValTok{65}\NormalTok{, }\DecValTok{82}\NormalTok{)}
\NormalTok{weight_g}
\end{Highlighting}
\end{Shaded}

A vector can also contain characters:

\begin{Shaded}
\begin{Highlighting}[]
\NormalTok{animals <-}\StringTok{ }\KeywordTok{c}\NormalTok{(}\StringTok{"mouse"}\NormalTok{, }\StringTok{"rat"}\NormalTok{, }\StringTok{"dog"}\NormalTok{)}
\NormalTok{animals}
\end{Highlighting}
\end{Shaded}

The quotes around ``mouse'', ``rat'', etc. are essential here. Without
the quotes R will assume there are objects called \texttt{mouse},
\texttt{rat} and \texttt{dog}. As these objects don't exist in R's
memory, there will be an error message.

There are many functions that allow you to inspect the content of a
vector. \texttt{length()} tells you how many elements are in a
particular vector:

\begin{Shaded}
\begin{Highlighting}[]
\KeywordTok{length}\NormalTok{(weight_g)}
\KeywordTok{length}\NormalTok{(animals)}
\end{Highlighting}
\end{Shaded}

An important feature of a vector, is that all of the elements are the
same type of data. The function \texttt{class()} indicates the class
(the type of element) of an object:

\begin{Shaded}
\begin{Highlighting}[]
\KeywordTok{class}\NormalTok{(weight_g)}
\KeywordTok{class}\NormalTok{(animals)}
\end{Highlighting}
\end{Shaded}

The function \texttt{str()} provides an overview of the structure of an
object and its elements. It is a useful function when working with large
and complex objects:

\begin{Shaded}
\begin{Highlighting}[]
\KeywordTok{str}\NormalTok{(weight_g)}
\KeywordTok{str}\NormalTok{(animals)}
\end{Highlighting}
\end{Shaded}

You can use the \texttt{c()} function to add other elements to your
vector:

\begin{Shaded}
\begin{Highlighting}[]
\NormalTok{weight_g <-}\StringTok{ }\KeywordTok{c}\NormalTok{(weight_g, }\DecValTok{90}\NormalTok{) }\CommentTok{# add to the end of the vector}
\NormalTok{weight_g <-}\StringTok{ }\KeywordTok{c}\NormalTok{(}\DecValTok{30}\NormalTok{, weight_g) }\CommentTok{# add to the beginning of the vector}
\NormalTok{weight_g}
\end{Highlighting}
\end{Shaded}

In the first line, we take the original vector \texttt{weight\_g}, add
the value \texttt{90} to the end of it, and save the result back into
\texttt{weight\_g}. Then we add the value \texttt{30} to the beginning,
again saving the result back into \texttt{weight\_g}.

We can do this over and over again to grow a vector, or assemble a
dataset. As we program, this may be useful to add results that we are
collecting or calculating.

We just saw 2 of the 6 main \textbf{atomic vector} types (or
\textbf{data types}) that R uses: \texttt{"character"} and
\texttt{"numeric"}. These are the basic building blocks that all R
objects are built from. The other 4 are:

\begin{itemize}
\tightlist
\item
  \texttt{"logical"} for \texttt{TRUE} and \texttt{FALSE} (the boolean
  data type)
\item
  \texttt{"integer"} for integer numbers (e.g., \texttt{2L}, the
  \texttt{L} indicates to R that it's an integer)
\item
  \texttt{"complex"} to represent complex numbers with real and
  imaginary parts (e.g., \texttt{1\ +\ 4i}) and that's all we're going
  to say about them
\item
  \texttt{"raw"} that we won't discuss further
\end{itemize}

\begin{quote}
Challenge

\begin{itemize}
\item
  We've seen that atomic vectors can be of type character, numeric,
  integer, and logical. But what happens if we try to mix these types in
  a single vector? 
\item
  What will happen in each of these examples? (hint: use
  \texttt{class()} to check the data type of your objects):

\begin{Shaded}
\begin{Highlighting}[]
\NormalTok{num_char <-}\StringTok{ }\KeywordTok{c}\NormalTok{(}\DecValTok{1}\NormalTok{, }\DecValTok{2}\NormalTok{, }\DecValTok{3}\NormalTok{, }\StringTok{'a'}\NormalTok{)}
\NormalTok{num_logical <-}\StringTok{ }\KeywordTok{c}\NormalTok{(}\DecValTok{1}\NormalTok{, }\DecValTok{2}\NormalTok{, }\DecValTok{3}\NormalTok{, }\OtherTok{TRUE}\NormalTok{)}
\NormalTok{char_logical <-}\StringTok{ }\KeywordTok{c}\NormalTok{(}\StringTok{'a'}\NormalTok{, }\StringTok{'b'}\NormalTok{, }\StringTok{'c'}\NormalTok{, }\OtherTok{TRUE}\NormalTok{)}
\NormalTok{tricky <-}\StringTok{ }\KeywordTok{c}\NormalTok{(}\DecValTok{1}\NormalTok{, }\DecValTok{2}\NormalTok{, }\DecValTok{3}\NormalTok{, }\StringTok{'4'}\NormalTok{)}
\end{Highlighting}
\end{Shaded}
\item
  Why do you think it happens? 
\item
  You've probably noticed that objects of different types get converted
  into a single, shared type within a vector. In R, we call converting
  objects from one class into another class \emph{coercion}. These
  conversions happen according to a hierarchy, whereby some types get
  preferentially coerced into other types. Can you draw a diagram that
  represents the hierarchy of how these data types are coerced? 
\end{itemize}
\end{quote}

\section{Subsetting vectors}\label{subsetting-vectors}

If we want to extract one or several values from a vector, we must
provide one or several indices in square brackets. For instance:

\begin{Shaded}
\begin{Highlighting}[]
\NormalTok{animals <-}\StringTok{ }\KeywordTok{c}\NormalTok{(}\StringTok{"mouse"}\NormalTok{, }\StringTok{"rat"}\NormalTok{, }\StringTok{"dog"}\NormalTok{, }\StringTok{"cat"}\NormalTok{)}
\NormalTok{animals[}\DecValTok{2}\NormalTok{]}
\end{Highlighting}
\end{Shaded}

\begin{verbatim}
#> [1] "rat"
\end{verbatim}

\begin{Shaded}
\begin{Highlighting}[]
\NormalTok{animals[}\KeywordTok{c}\NormalTok{(}\DecValTok{3}\NormalTok{, }\DecValTok{2}\NormalTok{)]}
\end{Highlighting}
\end{Shaded}

\begin{verbatim}
#> [1] "dog" "rat"
\end{verbatim}

We can also repeat the indices to create an object with more elements
than the original one:

\begin{Shaded}
\begin{Highlighting}[]
\NormalTok{more_animals <-}\StringTok{ }\NormalTok{animals[}\KeywordTok{c}\NormalTok{(}\DecValTok{1}\NormalTok{, }\DecValTok{2}\NormalTok{, }\DecValTok{3}\NormalTok{, }\DecValTok{2}\NormalTok{, }\DecValTok{1}\NormalTok{, }\DecValTok{4}\NormalTok{)]}
\NormalTok{more_animals}
\end{Highlighting}
\end{Shaded}

\begin{verbatim}
#> [1] "mouse" "rat"   "dog"   "rat"   "mouse" "cat"
\end{verbatim}

R indices start at 1. Programming languages like Fortran, MATLAB, Julia,
and R start counting at 1, because that's what human beings typically
do. Languages in the C family (including C++, Java, Perl, and Python)
count from 0 because that's simpler for computers to do.

\subsection{Conditional subsetting}\label{conditional-subsetting}

Another common way of subsetting is by using a logical vector.
\texttt{TRUE} will select the element with the same index, while
\texttt{FALSE} will not:

\begin{Shaded}
\begin{Highlighting}[]
\NormalTok{weight_g <-}\StringTok{ }\KeywordTok{c}\NormalTok{(}\DecValTok{21}\NormalTok{, }\DecValTok{34}\NormalTok{, }\DecValTok{39}\NormalTok{, }\DecValTok{54}\NormalTok{, }\DecValTok{55}\NormalTok{)}
\NormalTok{weight_g[}\KeywordTok{c}\NormalTok{(}\OtherTok{TRUE}\NormalTok{, }\OtherTok{FALSE}\NormalTok{, }\OtherTok{TRUE}\NormalTok{, }\OtherTok{TRUE}\NormalTok{, }\OtherTok{FALSE}\NormalTok{)]}
\end{Highlighting}
\end{Shaded}

\begin{verbatim}
#> [1] 21 39 54
\end{verbatim}

Typically, these logical vectors are not typed by hand, but are the
output of other functions or logical tests. For instance, if you wanted
to select only the values above 50:

\begin{Shaded}
\begin{Highlighting}[]
\NormalTok{weight_g >}\StringTok{ }\DecValTok{50}    \CommentTok{# will return logicals with TRUE for the indices that meet the condition}
\end{Highlighting}
\end{Shaded}

\begin{verbatim}
#> [1] FALSE FALSE FALSE  TRUE  TRUE
\end{verbatim}

\begin{Shaded}
\begin{Highlighting}[]
\NormalTok{## so we can use this to select only the values above 50}
\NormalTok{weight_g[weight_g >}\StringTok{ }\DecValTok{50}\NormalTok{]}
\end{Highlighting}
\end{Shaded}

\begin{verbatim}
#> [1] 54 55
\end{verbatim}

You can combine multiple tests using \texttt{\&} (both conditions are
true, AND) or \texttt{\textbar{}} (at least one of the conditions is
true, OR):

\begin{Shaded}
\begin{Highlighting}[]
\NormalTok{weight_g[weight_g <}\StringTok{ }\DecValTok{30} \NormalTok{|}\StringTok{ }\NormalTok{weight_g >}\StringTok{ }\DecValTok{50}\NormalTok{]}
\end{Highlighting}
\end{Shaded}

\begin{verbatim}
#> [1] 21 54 55
\end{verbatim}

\begin{Shaded}
\begin{Highlighting}[]
\NormalTok{weight_g[weight_g >=}\StringTok{ }\DecValTok{30} \NormalTok{&}\StringTok{ }\NormalTok{weight_g ==}\StringTok{ }\DecValTok{21}\NormalTok{]}
\end{Highlighting}
\end{Shaded}

\begin{verbatim}
#> numeric(0)
\end{verbatim}

Here, \texttt{\textless{}} stands for ``less than'',
\texttt{\textgreater{}} for ``greater than'', \texttt{\textgreater{}=}
for ``greater than or equal to'', and \texttt{==} for ``equal to''. The
double equal sign \texttt{==} is a test for numerical equality between
the left and right hand sides, and should not be confused with the
single \texttt{=} sign, which performs variable assignment (similar to
\texttt{\textless{}-}).

A common task is to search for certain strings in a vector. One could
use the ``or'' operator \texttt{\textbar{}} to test for equality to
multiple values, but this can quickly become tedious. The function
\texttt{\%in\%} allows you to test if any of the elements of a search
vector are found:

\begin{Shaded}
\begin{Highlighting}[]
\NormalTok{animals <-}\StringTok{ }\KeywordTok{c}\NormalTok{(}\StringTok{"mouse"}\NormalTok{, }\StringTok{"rat"}\NormalTok{, }\StringTok{"dog"}\NormalTok{, }\StringTok{"cat"}\NormalTok{)}
\NormalTok{animals[animals ==}\StringTok{ "cat"} \NormalTok{|}\StringTok{ }\NormalTok{animals ==}\StringTok{ "rat"}\NormalTok{] }\CommentTok{# returns both rat and cat}
\end{Highlighting}
\end{Shaded}

\begin{verbatim}
#> [1] "rat" "cat"
\end{verbatim}

\begin{Shaded}
\begin{Highlighting}[]
\NormalTok{animals %in%}\StringTok{ }\KeywordTok{c}\NormalTok{(}\StringTok{"rat"}\NormalTok{, }\StringTok{"cat"}\NormalTok{, }\StringTok{"dog"}\NormalTok{, }\StringTok{"duck"}\NormalTok{, }\StringTok{"goat"}\NormalTok{)}
\end{Highlighting}
\end{Shaded}

\begin{verbatim}
#> [1] FALSE  TRUE  TRUE  TRUE
\end{verbatim}

\begin{Shaded}
\begin{Highlighting}[]
\NormalTok{animals[animals %in%}\StringTok{ }\KeywordTok{c}\NormalTok{(}\StringTok{"rat"}\NormalTok{, }\StringTok{"cat"}\NormalTok{, }\StringTok{"dog"}\NormalTok{, }\StringTok{"duck"}\NormalTok{, }\StringTok{"goat"}\NormalTok{)]}
\end{Highlighting}
\end{Shaded}

\begin{verbatim}
#> [1] "rat" "dog" "cat"
\end{verbatim}

\begin{quote}
Challenge

\begin{itemize}
\tightlist
\item
  Can you figure out why \texttt{"four"\ \textgreater{}\ "five"} returns
  \texttt{TRUE}?
\end{itemize}
\end{quote}

\section{Missing data}\label{missing-data}

As R was designed to analyze datasets, it includes the concept of
missing data (which is uncommon in other programming languages). Missing
data are represented in vectors as \texttt{NA}.

When doing operations on numbers, most functions will return \texttt{NA}
if the data you are working with include missing values. This feature
makes it harder to overlook the cases where you are dealing with missing
data. You can add the argument \texttt{na.rm=TRUE} to calculate the
result while ignoring the missing values.

\begin{Shaded}
\begin{Highlighting}[]
\NormalTok{heights <-}\StringTok{ }\KeywordTok{c}\NormalTok{(}\DecValTok{2}\NormalTok{, }\DecValTok{4}\NormalTok{, }\DecValTok{4}\NormalTok{, }\OtherTok{NA}\NormalTok{, }\DecValTok{6}\NormalTok{)}
\KeywordTok{mean}\NormalTok{(heights)}
\KeywordTok{max}\NormalTok{(heights)}
\KeywordTok{mean}\NormalTok{(heights, }\DataTypeTok{na.rm =} \OtherTok{TRUE}\NormalTok{)}
\KeywordTok{max}\NormalTok{(heights, }\DataTypeTok{na.rm =} \OtherTok{TRUE}\NormalTok{)}
\end{Highlighting}
\end{Shaded}

If your data include missing values, you may want to become familiar
with the functions \texttt{is.na()}, \texttt{na.omit()}, and
\texttt{complete.cases()}. See below for examples.

\begin{Shaded}
\begin{Highlighting}[]
\NormalTok{## Extract those elements which are not missing values.}
\NormalTok{heights[!}\KeywordTok{is.na}\NormalTok{(heights)]}

\NormalTok{## Returns the object with incomplete cases removed. The returned object is atomic.}
\KeywordTok{na.omit}\NormalTok{(heights)}

\NormalTok{## Extract those elements which are complete cases.}
\NormalTok{heights[}\KeywordTok{complete.cases}\NormalTok{(heights)]}
\end{Highlighting}
\end{Shaded}

\begin{quote}
Challenge

\begin{enumerate}
\def\labelenumi{\arabic{enumi}.}
\item
  Using this vector of length measurements, create a new vector with the
  NAs removed.

\begin{Shaded}
\begin{Highlighting}[]
\NormalTok{lengths <-}\StringTok{ }\KeywordTok{c}\NormalTok{(}\DecValTok{10}\NormalTok{,}\DecValTok{24}\NormalTok{,}\OtherTok{NA}\NormalTok{,}\DecValTok{18}\NormalTok{,}\OtherTok{NA}\NormalTok{,}\DecValTok{20}\NormalTok{)}
\end{Highlighting}
\end{Shaded}
\item
  Use the function \texttt{median()} to calculate the median of the
  \texttt{lengths} vector.
\end{enumerate}
\end{quote}

\section{Common R Data Structures}\label{common-r-data-structures}

Vectors are one of the many \textbf{data structures} that R uses. Other
important ones are matrices (\texttt{matrix}), s (\texttt{data.frame}),
lists (\texttt{list}), and factors (\texttt{factor}).

\subsection{Matrix}\label{matrix}

If we arrange data elements of a vector in a two-dimensional rectangular
layout we have a matrix. To construct a matrix, we use a function
conveniently called \texttt{matrix()}.

\begin{Shaded}
\begin{Highlighting}[]
\NormalTok{y <-}\StringTok{ }\KeywordTok{matrix}\NormalTok{(}\DecValTok{1}\NormalTok{:}\DecValTok{20}\NormalTok{, }\DataTypeTok{nrow=}\DecValTok{5}\NormalTok{,}\DataTypeTok{ncol=}\DecValTok{4}\NormalTok{) }\CommentTok{# generates 5 x 4 numeric matrix}
\end{Highlighting}
\end{Shaded}

Subset a matrix with {[}row \texttt{,} column{]}:

\begin{Shaded}
\begin{Highlighting}[]
\NormalTok{y[,}\DecValTok{4}\NormalTok{]       }\CommentTok{# 4th column of matrix}
\NormalTok{y[}\DecValTok{3}\NormalTok{,]       }\CommentTok{# 3rd row of matrix}
\NormalTok{y[}\DecValTok{2}\NormalTok{:}\DecValTok{4}\NormalTok{,}\DecValTok{1}\NormalTok{:}\DecValTok{3}\NormalTok{]  }\CommentTok{# rows 2,3,4 of columns 1,2,3}
\end{Highlighting}
\end{Shaded}

\begin{quote}
Not surprisingly 2-dimensional matrices play an important role when
working with raster data. We will come back to that at a later time.
\end{quote}

\subsection{List}\label{list}

Lists can have elements of any type. Here is how we construct lists. You
may have guessed that to construct a list, we use the \texttt{list()}
function:

\begin{Shaded}
\begin{Highlighting}[]
\NormalTok{myl <-}\StringTok{ }\KeywordTok{list}\NormalTok{(}\DataTypeTok{name=}\StringTok{"Sue"}\NormalTok{, }\DataTypeTok{mynumbers=}\NormalTok{a, }\DataTypeTok{mymatrix=}\NormalTok{y, }\DataTypeTok{age=}\FloatTok{5.3}\NormalTok{) }\CommentTok{# example of a list with 4 components}

\NormalTok{myl[[}\DecValTok{2}\NormalTok{]] }\CommentTok{# 2nd component of the list}
\NormalTok{myl[[}\StringTok{"mynumbers"}\NormalTok{]] }\CommentTok{# component named mynumbers in list}
\end{Highlighting}
\end{Shaded}

\subsection{Data frame}\label{data-frame}

Data frames in R are a special case of lists, as they can have elements
of any type, but they have to \textbf{all be of the same length}.

A data frame is the most common way of storing tabular data in R and
something you will likely deal with a lot. For example, attribute data
for vector based spatial objects in R are stored as a data frame. You
can really think of it as a table or a spreadsheet. It a 2-dimensional
structure and columns can be of different element types. It is a special
case of a list, where each of the (list) elements has the same length.

Here is how you could construct a data frame.

\begin{Shaded}
\begin{Highlighting}[]
\NormalTok{mydf <-}\StringTok{ }\KeywordTok{data.frame}\NormalTok{(}\DataTypeTok{ID=}\KeywordTok{c}\NormalTok{(}\DecValTok{1}\NormalTok{:}\DecValTok{4}\NormalTok{),}
                   \DataTypeTok{Color=}\KeywordTok{c}\NormalTok{(}\StringTok{"red"}\NormalTok{, }\StringTok{"white"}\NormalTok{, }\StringTok{"red"}\NormalTok{, }\OtherTok{NA}\NormalTok{),}
                   \DataTypeTok{Passed=}\KeywordTok{c}\NormalTok{(}\OtherTok{TRUE}\NormalTok{,}\OtherTok{TRUE}\NormalTok{,}\OtherTok{TRUE}\NormalTok{,}\OtherTok{FALSE}\NormalTok{),}
                   \DataTypeTok{Weight=}\KeywordTok{c}\NormalTok{(}\DecValTok{99}\NormalTok{, }\DecValTok{54}\NormalTok{, }\DecValTok{85}\NormalTok{, }\DecValTok{70}\NormalTok{),}
                   \DataTypeTok{Height=}\KeywordTok{c}\NormalTok{(}\FloatTok{1.78}\NormalTok{, }\FloatTok{1.67}\NormalTok{, }\FloatTok{1.82}\NormalTok{, }\FloatTok{1.59}\NormalTok{))}

\NormalTok{mydf}
\end{Highlighting}
\end{Shaded}

We will go into more detail about data frames and factors later on. For
now, try the following:

\begin{quote}
\section{Challenge}\label{challenge}

\begin{enumerate}
\def\labelenumi{\arabic{enumi}.}
\tightlist
\item
  Create a data frame that holds the following information for yourself,
  your right and your left neighbor:
\end{enumerate}

\begin{itemize}
\tightlist
\item
  first name
\item
  last name
\item
  lucky number
\end{itemize}

\begin{enumerate}
\def\labelenumi{\arabic{enumi}.}
\setcounter{enumi}{1}
\item
  There are a few mistakes in this hand-crafted \texttt{data.frame}, can
  you spot and fix them? Don't hesitate to experiment!

\begin{Shaded}
\begin{Highlighting}[]
\NormalTok{animal_data <-}\StringTok{ }\KeywordTok{data.frame}\NormalTok{(}\DataTypeTok{animal=}\KeywordTok{c}\NormalTok{(}\StringTok{"dog"}\NormalTok{, }\StringTok{"cat"}\NormalTok{, }\StringTok{"sea cucumber"}\NormalTok{, }\StringTok{"sea urchin"}\NormalTok{),}
                          \DataTypeTok{feel=}\KeywordTok{c}\NormalTok{(}\StringTok{"furry"}\NormalTok{, }\StringTok{"squishy"}\NormalTok{, }\StringTok{"spiny"}\NormalTok{),}
                          \DataTypeTok{weight=}\KeywordTok{c}\NormalTok{(}\DecValTok{45}\NormalTok{, }\DecValTok{8} \FloatTok{1.1}\NormalTok{, }\FloatTok{0.8}\NormalTok{))}
\end{Highlighting}
\end{Shaded}
\end{enumerate}
\end{quote}

\begin{quote}
\textless{}!--- Answers

\begin{figure}
\includegraphics[width=0.4\linewidth]{img/installpckg1} \includegraphics[width=0.4\linewidth]{img/installpckg2} \caption{How to install an R package with the RStudio interface}\label{fig:install-packages}
\end{figure}
\end{quote}

\begin{enumerate}
\def\labelenumi{\arabic{enumi}.}
\setcounter{enumi}{1}
\tightlist
\item
  You can install from the R console like this:
\end{enumerate}

\begin{Shaded}
\begin{Highlighting}[]
\KeywordTok{install.packages}\NormalTok{(}\StringTok{"name_of_package_to_install"}\NormalTok{, }\DataTypeTok{dependencies =} \OtherTok{TRUE}\NormalTok{)}
\end{Highlighting}
\end{Shaded}

\subsection{Make use of the installed
packages}\label{make-use-of-the-installed-packages}

In order to actually use commands from the installed packages you also
will need to load the installed packages. This can be automated
(whenever you launch R it will also load the libraries for you - see for
example \href{http://stackoverflow.com/a/14238658/2630957}{here}) or
otherwise you need to sumbit a command:

\begin{Shaded}
\begin{Highlighting}[]
\KeywordTok{library}\NormalTok{(name_of_package_to_install)}
\end{Highlighting}
\end{Shaded}

or

\begin{Shaded}
\begin{Highlighting}[]
\KeywordTok{require}\NormalTok{(name_of_package_to_install)}
\end{Highlighting}
\end{Shaded}

The difference between the two is that \texttt{library} will result in
an error, if the library does not exist, whereas \texttt{require} will
result in a warning.

\begin{quote}
\section{Challenge 3}\label{challenge-3}

\begin{enumerate}
\def\labelenumi{\arabic{enumi}.}
\tightlist
\item
  Google for an R package that might be of interest for your work.
\item
  Install and load it into R.
\end{enumerate}
\end{quote}

\section{Seeking help}\label{seeking-help}

\subsection{Use the built-in RStudio help
interface}\label{use-the-built-in-rstudio-help-interface}

\begin{figure}
\includegraphics[width=0.8\linewidth]{img/rstudiohelp} \caption{The RStudio help interface}\label{fig:use-help}
\end{figure}

One of the most immediate ways to get help, is to use the RStudio help
interface. This panel by default can be found at the lower right hand
panel of RStudio. As seen in the screenshot, by typing the word
``Mean'', RStudio tries to also give a number of suggestions that you
might be interested in. The description is then shown in the display
window.

\subsection{I know the name of the function, but I'm not sure how to use
it}\label{i-know-the-name-of-the-function-but-im-not-sure-how-to-use-it}

If you need help with a specific function, let's say \texttt{barplot()},
you can type:

\begin{Shaded}
\begin{Highlighting}[]
\NormalTok{?barplot}
\end{Highlighting}
\end{Shaded}

If you just need to remind yourself of the names of the arguments, you
can use:

\begin{Shaded}
\begin{Highlighting}[]
\KeywordTok{args}\NormalTok{(lm)}
\end{Highlighting}
\end{Shaded}

\subsection{There must be a function to do X but I don't know which
one\ldots{}}\label{there-must-be-a-function-to-do-x-but-i-dont-know-which-one}

If you are looking for a function to do a particular task, you can use
the \texttt{help.search()} function, which is called by the double
question mark \texttt{??}. However, this only looks through the
installed packages for help pages with a match to your search request

\begin{Shaded}
\begin{Highlighting}[]
\NormalTok{??kruskal}
\end{Highlighting}
\end{Shaded}

If you can't find what you are looking for, you can use the
\href{http://www.rdocumentation.org}{rdocumentation.org} website that
searches through the help files across all packages available.

Finally, a generic Google or internet search ``R
\textless{}task\textgreater{}'' will often either send you to the
appropriate package documentation or a helpful forum where someone else
has already asked your question.

\subsection{I am stuck\ldots{} I get an error message that I don't
understand}\label{i-am-stuck-i-get-an-error-message-that-i-dont-understand}

Start by googling the error message. However, this doesn't always work
very well because often, package developers rely on the error catching
provided by R. You end up with general error messages that might not be
very helpful to diagnose a problem (e.g. ``subscript out of bounds'').
If the message is very generic, you might also include the name of the
function or package you're using in your query.

However, you should check Stack Overflow. Search using the
\texttt{{[}r{]}} tag. Most questions have already been answered, but the
challenge is to use the right words in the search to find the answers:
\url{http://stackoverflow.com/questions/tagged/r}

The
\href{http://cran.r-project.org/doc/manuals/R-intro.pdf}{Introduction to
R} can also be dense for people with little programming experience but
it is a good place to understand the underpinnings of the R language.

The \href{http://cran.r-project.org/doc/FAQ/R-FAQ.html}{R FAQ} is dense
and technical but it is full of useful information.

\subsection{How to ask for help}\label{how-to-ask-for-help}

The key to receiving help from someone is for them to rapidly grasp your
problem. You should make it as easy as possible to pinpoint where the
issue might be.

Try to use the correct words to describe your problem. For instance, a
package is not the same thing as a library. Most people will understand
what you meant, but others have really strong feelings about the
difference in meaning. The key point is that it can make things
confusing for people trying to help you. Be as precise as possible when
describing your problem.

If possible, try to reduce what doesn't work to a simple
\emph{reproducible example}. If you can reproduce the problem using a
very small data frame instead of your 50,000 rows and 10,000 columns
one, provide the small one with the description of your problem. When
appropriate, try to generalize what you are doing so even people who are
not in your field can understand the question. For instance instead of
using a subset of your real dataset, create a small (3 columns, 5 rows)
generic one. For more information on how to write a reproducible example
see \href{http://adv-r.had.co.nz/Reproducibility.html}{this article by
Hadley Wickham}.

To share an object with someone else, if it's relatively small, you can
use the function \texttt{dput()}. It will output R code that can be used
to recreate the exact same object as the one in memory:

\begin{Shaded}
\begin{Highlighting}[]
\KeywordTok{dput}\NormalTok{(}\KeywordTok{head}\NormalTok{(iris)) }\CommentTok{# iris is an example data frame that comes with R and head() is a function that returns the first part of the data frame}
\end{Highlighting}
\end{Shaded}

\begin{verbatim}
#> structure(list(Sepal.Length = c(5.1, 4.9, 4.7, 4.6, 5, 5.4), 
#>     Sepal.Width = c(3.5, 3, 3.2, 3.1, 3.6, 3.9), Petal.Length = c(1.4, 
#>     1.4, 1.3, 1.5, 1.4, 1.7), Petal.Width = c(0.2, 0.2, 0.2, 
#>     0.2, 0.2, 0.4), Species = structure(c(1L, 1L, 1L, 1L, 1L, 
#>     1L), .Label = c("setosa", "versicolor", "virginica"), class = "factor")), .Names = c("Sepal.Length", 
#> "Sepal.Width", "Petal.Length", "Petal.Width", "Species"), row.names = c(NA, 
#> 6L), class = "data.frame")
\end{verbatim}

If the object is larger, provide either the raw file (i.e., your CSV
file) with your script up to the point of the error (and after removing
everything that is not relevant to your issue). Alternatively, in
particular if your question is not related to a data frame, you can save
any R object to a file:

\begin{Shaded}
\begin{Highlighting}[]
\KeywordTok{saveRDS}\NormalTok{(iris, }\DataTypeTok{file=}\StringTok{"/tmp/iris.rds"}\NormalTok{)}
\end{Highlighting}
\end{Shaded}

The content of this file is however not human readable and cannot be
posted directly on Stack Overflow. Instead, it can be sent to someone by
email who can read it with the \texttt{readRDS()} command (here it is
assumed that the downloaded file is in a \texttt{Downloads} folder in
the user's home directory):

\begin{Shaded}
\begin{Highlighting}[]
\NormalTok{some_data <-}\StringTok{ }\KeywordTok{readRDS}\NormalTok{(}\DataTypeTok{file=}\StringTok{"~/Downloads/iris.rds"}\NormalTok{)}
\end{Highlighting}
\end{Shaded}

Last, but certainly not least, \textbf{always include the output of
\texttt{sessionInfo()}} as it provides critical information about your
platform, the versions of R and the packages that you are using, and
other information that can be very helpful to understand your problem.

\begin{Shaded}
\begin{Highlighting}[]
\KeywordTok{sessionInfo}\NormalTok{()}
\end{Highlighting}
\end{Shaded}

\begin{verbatim}
#> R version 3.4.1 (2017-06-30)
#> Platform: x86_64-apple-darwin15.6.0 (64-bit)
#> Running under: macOS Sierra 10.12.6
#> 
#> Matrix products: default
#> BLAS: /Library/Frameworks/R.framework/Versions/3.4/Resources/lib/libRblas.0.dylib
#> LAPACK: /Library/Frameworks/R.framework/Versions/3.4/Resources/lib/libRlapack.dylib
#> 
#> locale:
#> [1] en_US.UTF-8/en_US.UTF-8/en_US.UTF-8/C/en_US.UTF-8/en_US.UTF-8
#> 
#> attached base packages:
#> [1] stats     graphics  grDevices utils     datasets  methods   base     
#> 
#> loaded via a namespace (and not attached):
#>  [1] compiler_3.4.1  backports_1.0.5 bookdown_0.4.2  magrittr_1.5   
#>  [5] rprojroot_1.2   tools_3.4.1     htmltools_0.3.6 rstudioapi_0.6 
#>  [9] yaml_2.1.14     Rcpp_0.12.12    stringi_1.1.5   rmarkdown_1.6  
#> [13] knitr_1.17      stringr_1.2.0   digest_0.6.12   evaluate_0.10.1
\end{verbatim}

\subsection{Where to ask for help?}\label{where-to-ask-for-help}

\begin{itemize}
\tightlist
\item
  The person sitting next to you during the workshop. Don't hesitate to
  talk to your neighbor during the workshop, compare your answers, and
  ask for help. You might also be interested in organizing regular
  meetings following the workshop to keep learning from each other.
\item
  Your friendly colleagues: if you know someone with more experience
  than you, they might be able and willing to help you.
\item
  \href{http://stackoverflow.com/questions/tagged/r}{Stack Overflow}: if
  your question hasn't been answered before and is well crafted, chances
  are you will get an answer in less than 5 min. Remember to follow
  their guidelines on
  \href{http://stackoverflow.com/help/how-to-ask}{how to ask a good
  question}.
\item
  The \href{https://stat.ethz.ch/mailman/listinfo/r-help}{R-help mailing
  list}: it is read by a lot of people (including most of the R core
  team), a lot of people post to it, but the tone can be pretty dry, and
  it is not always very welcoming to new users. If your question is
  valid, you are likely to get an answer very fast but don't expect that
  it will come with smiley faces. Also, here more than anywhere else, be
  sure to use correct vocabulary (otherwise you might get an answer
  pointing to the misuse of your words rather than answering your
  question). You will also have more success if your question is about a
  base function rather than a specific package.
\item
  If your question is about a specific package, see if there is a
  mailing list for it. Usually it's included in the DESCRIPTION file of
  the package that can be accessed using
  \texttt{packageDescription("name-of-package")}. You may also want to
  try to email the author of the package directly, or open an issue on
  the code repository (e.g., GitHub).
\item
  There are also some topic-specific mailing lists (GIS, phylogenetics,
  etc\ldots{}), the complete list is
  \href{http://www.r-project.org/mail.html}{here}.
\end{itemize}

\subsection{Resources on getting help}\label{resources-on-getting-help}

\begin{itemize}
\tightlist
\item
  The \href{http://www.r-project.org/posting-guide.html}{Posting Guide}
  for the R mailing lists.
\item
  \href{http://blog.revolutionanalytics.com/2014/01/how-to-ask-for-r-help.html}{How
  to ask for R help} useful guidelines
\item
  \href{http://codeblog.jonskeet.uk/2010/08/29/writing-the-perfect-question/}{This
  blog post by Jon Skeet} has quite comprehensive advice on how to ask
  programming questions.
\item
  The \href{https://cran.rstudio.com/web/packages/reprex/}{reprex}
  package is very helpful to create reproducible examples when asking
  for help. The {[}rOpenSci community call ``How to ask questions so
  they get answered''{]},
  \href{https://github.com/ropensci/commcalls/issues/14}{Github link}
  and \href{https://vimeo.com/208749032}{video recording} includes a
  presentation of the reprex package and of its philosophy.
\end{itemize}

\chapter{Working with tabular data in R}\label{data}

\begin{quote}
Learning Objectives

\begin{itemize}
\tightlist
\item
  Load external data from a .csv file into a data frame in R with
  \texttt{read.csv()}
\item
  Find basic properties of a data frames including size, class or type
  of the columns, names of rows and columns by using \texttt{str()},
  \texttt{nrow()}, \texttt{ncol()}, \texttt{dim()}, \texttt{length()} ,
  \texttt{colnames()}, \texttt{rownames()}
\item
  Use \texttt{head()} and \texttt{tail()} to inspect rows of a data
  frame.
\item
  Generate summary statistics for a data frame
\item
  Use indexing to select rows and columns
\item
  Use logical conditions to select rows and columns
\item
  Add columns and rows to a data frame
\item
  Manipulate categorical data with \texttt{factors}, \texttt{levels()}
  and \texttt{as.character()}
\item
  Change how character strings are handled in a data frame.
\item
  Format dates in R and calculate time differences
\item
  Use \texttt{df\$new\_col\ \textless{}-\ new\_col} to add a new column
  to a data frame.
\item
  Use \texttt{cbind()} to add a new column to a data frame.
\item
  Use \texttt{rbind()} to add a new row to a data frame.
\item
  Use \texttt{na.omit()} to remove rows from a data frame with
  \texttt{NA} values.
\end{itemize}
\end{quote}

\begin{center}\rule{0.5\linewidth}{\linethickness}\end{center}

\section{Loading tabular data}\label{loading-tabular-data}

One the most common ways of getting data into R is to read in a table.
And -- you guessed it -- we read it into a data frame! We will take a
simple CSV file as example.
\href{https://support.bigcommerce.com/articles/Public/What-is-a-CSV-file-and-how-do-I-save-my-spreadsheet-as-one}{What
is a CSV file?}

You may know about \href{https://openpolicing.stanford.edu}{the Stanford
Open Policing Project} and we will be working a sample dataset from
their repository (\url{https://openpolicing.stanford.edu/data/}). It
contains information about traffic stops for blacks and whites in the
state of Mississippi during January 2013 to mid-July of 2016.

We are going to use the R function \texttt{download.file()} to download
the CSV file that contains the traffic stop data, and we will use
\texttt{read.csv()} to load into memory the content of the CSV file as
an object of class \texttt{data.frame}.

Download the traffic stop data from here: \ldots{}..

To download the data into the \texttt{data/} subdirectory, run the
following:

\begin{Shaded}
\begin{Highlighting}[]
\KeywordTok{download.file}\NormalTok{(}\StringTok{"..."}\NormalTok{,}
              \StringTok{"data/MS_trafficstops_bw.csv"}\NormalTok{)}
\end{Highlighting}
\end{Shaded}

You are now ready to load the data:

\begin{Shaded}
\begin{Highlighting}[]
\NormalTok{trafficstops <-}\StringTok{ }\KeywordTok{read.csv}\NormalTok{(}\StringTok{'data/MS_trafficstops_bw.csv'}\NormalTok{)}
\end{Highlighting}
\end{Shaded}

This statement doesn't produce any output because, as you might recall,
assignments don't display anything. If we want to check that our data
has been loaded, we can print the variable's value:
\texttt{trafficstops}.

Wow\ldots{} that was a lot of output. At least it means the data loaded
properly. Let's check the top (the first 6 lines) of this data frame
using the function \texttt{head()}:

\begin{Shaded}
\begin{Highlighting}[]
\KeywordTok{head}\NormalTok{(trafficstops)}
\end{Highlighting}
\end{Shaded}

\begin{verbatim}
#>              id state  stop_date       county_name county_fips
#> 1 MS-2013-00001    MS 2013-01-01      Jones County       28067
#> 2 MS-2013-00002    MS 2013-01-01 Lauderdale County       28075
#> 3 MS-2013-00003    MS 2013-01-01       Pike County       28113
#> 4 MS-2013-00004    MS 2013-01-01    Hancock County       28045
#> 5 MS-2013-00005    MS 2013-01-01     Holmes County       28051
#> 6 MS-2013-00006    MS 2013-01-01    Jackson County       28059
#>            police_department driver_gender driver_birthdate driver_race
#> 1 Mississippi Highway Patrol             M       1950-06-14       Black
#> 2 Mississippi Highway Patrol             M       1967-04-06       Black
#> 3 Mississippi Highway Patrol             M       1974-04-15       Black
#> 4 Mississippi Highway Patrol             M       1981-03-23       White
#> 5 Mississippi Highway Patrol             M       1992-08-03       White
#> 6 Mississippi Highway Patrol             F       1960-05-02       White
#>                                                 violation_raw officer_id
#> 1                     Seat belt not used properly as required       J042
#> 2                                            Careless driving       B026
#> 3 Speeding - Regulated or posted speed limit and actual speed       M009
#> 4 Speeding - Regulated or posted speed limit and actual speed       K035
#> 5 Speeding - Regulated or posted speed limit and actual speed       D028
#> 6 Speeding - Regulated or posted speed limit and actual speed       K023
\end{verbatim}

\section{\texorpdfstring{Inspecting \texttt{data.frame}
Objects}{Inspecting data.frame Objects}}\label{inspecting-data.frame-objects}

As you may recall, a data frame in R is a special case of a list, and a
representation of data where the columns are vectors that all have the
same length. Because the columns are vectors, they all contain the same
type of data (e.g., characters, integers, factors, etc.).

We can see this when inspecting the structure of a data frame with the
function \texttt{str()}:

\begin{Shaded}
\begin{Highlighting}[]
\KeywordTok{str}\NormalTok{(trafficstops)}
\end{Highlighting}
\end{Shaded}

\begin{verbatim}
#> 'data.frame':    211211 obs. of  11 variables:
#>  $ id               : Factor w/ 211211 levels "MS-2013-00001",..: 1 2 3 4 5 6 7 8 9 10 ...
#>  $ state            : Factor w/ 1 level "MS": 1 1 1 1 1 1 1 1 1 1 ...
#>  $ stop_date        : Factor w/ 1288 levels "2013-01-01","2013-01-02",..: 1 1 1 1 1 1 1 1 1 1 ...
#>  $ county_name      : Factor w/ 82 levels "Adams County",..: 34 38 57 23 26 30 30 22 26 26 ...
#>  $ county_fips      : int  28067 28075 28113 28045 28051 28059 28059 28043 28051 28051 ...
#>  $ police_department: Factor w/ 1 level "Mississippi Highway Patrol": 1 1 1 1 1 1 1 1 1 1 ...
#>  $ driver_gender    : Factor w/ 3 levels "","F","M": 3 3 3 3 3 2 2 2 3 3 ...
#>  $ driver_birthdate : Factor w/ 21423 levels "","1930-01-11",..: 3558 9575 12137 14670 18820 7061 4504 19135 2755 15878 ...
#>  $ driver_race      : Factor w/ 3 levels "","Black","White": 2 2 2 3 3 3 3 3 3 3 ...
#>  $ violation_raw    : Factor w/ 19 levels "??","Careless driving",..: 17 2 19 19 19 19 19 19 19 19 ...
#>  $ officer_id       : Factor w/ 897 levels "","A003","A004",..: 519 52 635 560 212 550 559 205 723 723 ...
\end{verbatim}

We already saw how the functions \texttt{head()} and \texttt{str()} can
be useful to check the content and the structure of a data frame. Here
is a non-exhaustive list of functions to get a sense of the
content/structure of the data. Let's try them out!

\begin{itemize}
\tightlist
\item
  Size:

  \begin{itemize}
  \tightlist
  \item
    \texttt{dim(trafficstops)} - returns a vector with the number of
    rows in the first element, and the number of columns as the second
    element (the \textbf{dim}ensions of the object)
  \item
    \texttt{nrow(trafficstops)} - returns the number of rows
  \item
    \texttt{ncol(trafficstops)} - returns the number of columns
  \item
    \texttt{length(trafficstops)} - returns number of columns
  \end{itemize}
\item
  Content:

  \begin{itemize}
  \tightlist
  \item
    \texttt{head(trafficstops)} - shows the first 6 rows
  \item
    \texttt{tail(trafficstops)} - shows the last 6 rows
  \end{itemize}
\item
  Names:

  \begin{itemize}
  \tightlist
  \item
    \texttt{names(trafficstops)} - returns the column names (synonym of
    \texttt{colnames()} for \texttt{data.frame} objects)
  \item
    \texttt{rownames(trafficstops)} - returns the row names
  \end{itemize}
\item
  Summary:

  \begin{itemize}
  \tightlist
  \item
    \texttt{str(trafficstops)} - structure of the object and information
    about the class, length and content of each column
  \item
    \texttt{summary(trafficstops)} - summary statistics for each column
  \end{itemize}
\end{itemize}

Note: most of these functions are ``generic'', they can be used on other
types of objects besides \texttt{data.frame}.

\begin{quote}
Challenge

Based on the output of \texttt{str(trafficstops)}, can you answer the
following questions?

\begin{itemize}
\tightlist
\item
  What is the class of the object \texttt{trafficstops}?
\item
  How many rows and how many columns are in this object?
\item
  How many counties have been recorded in this dataset?
\end{itemize}
\end{quote}

\section{Indexing and subsetting data
frames}\label{indexing-and-subsetting-data-frames}

Our gapminder data frame has rows and columns (it has 2 dimensions), if
we want to extract some specific data from it, we need to specify the
``coordinates'' we want from it. Row numbers come first, followed by
column numbers. However, note that different ways of specifying these
coordinates lead to results with different classes.

\begin{Shaded}
\begin{Highlighting}[]
\NormalTok{trafficstops[}\DecValTok{1}\NormalTok{, }\DecValTok{1}\NormalTok{]   }\CommentTok{# first element in the first column of the data frame (as a vector)}
\NormalTok{trafficstops[}\DecValTok{1}\NormalTok{, }\DecValTok{6}\NormalTok{]   }\CommentTok{# first element in the 6th column (as a vector)}
\NormalTok{trafficstops[, }\DecValTok{1}\NormalTok{]    }\CommentTok{# first column in the data frame (as a vector)}
\NormalTok{trafficstops[}\DecValTok{1}\NormalTok{]      }\CommentTok{# first column in the data frame (as a data.frame)}
\NormalTok{trafficstops[}\DecValTok{1}\NormalTok{:}\DecValTok{3}\NormalTok{, }\DecValTok{7}\NormalTok{] }\CommentTok{# first three elements in the 7th column (as a vector)}
\NormalTok{trafficstops[}\DecValTok{3}\NormalTok{, ]    }\CommentTok{# the 3rd element for all columns (as a data.frame)}
\NormalTok{head_trafficstops <-}\StringTok{ }\NormalTok{trafficstops[}\DecValTok{1}\NormalTok{:}\DecValTok{6}\NormalTok{, ] }\CommentTok{# equivalent to head(gapminder)}
\end{Highlighting}
\end{Shaded}

\texttt{:} is a special function that creates numeric vectors of
integers in increasing or decreasing order, test \texttt{1:10} and
\texttt{10:1} for instance.

You can also exclude certain parts of a data frame using the
``\texttt{-}'' sign:

\begin{Shaded}
\begin{Highlighting}[]
\NormalTok{trafficstops[,-}\DecValTok{1}\NormalTok{]          }\CommentTok{# The whole data frame, except the first column}
\NormalTok{trafficstops[-}\KeywordTok{c}\NormalTok{(}\DecValTok{7}\NormalTok{:}\DecValTok{211211}\NormalTok{),] }\CommentTok{# Equivalent to head(trafficstops)}
\end{Highlighting}
\end{Shaded}

As well as using numeric values to subset a \texttt{data.frame} (or
\texttt{matrix}), columns can be called by name, using one of the four
following notations:

\begin{Shaded}
\begin{Highlighting}[]
\NormalTok{trafficstops[}\StringTok{"violation_raw"}\NormalTok{]       }\CommentTok{# Result is a data.frame}
\NormalTok{trafficstops[, }\StringTok{"violation_raw"}\NormalTok{]     }\CommentTok{# Result is a vector}
\NormalTok{trafficstops[[}\StringTok{"violation_raw"}\NormalTok{]]     }\CommentTok{# Result is a vector}
\NormalTok{trafficstops$violation_raw          }\CommentTok{# Result is a vector}
\end{Highlighting}
\end{Shaded}

For our purposes, the last three notations are equivalent. RStudio knows
about the columns in your data frame, so you can take advantage of the
autocompletion feature to get the full and correct column name.

\begin{quote}
Challenge

\begin{enumerate}
\def\labelenumi{\arabic{enumi}.}
\item
  Create a \texttt{data.frame} (\texttt{trafficstops\_200}) containing
  only the observations from row 200 of the \texttt{trafficstops}
  dataset.
\item
  Notice how \texttt{nrow()} gave you the number of rows in a
  \texttt{data.frame}?

  \begin{itemize}
  \tightlist
  \item
    Use that number to pull out just that last row in the data frame.
  \item
    Compare that with what you see as the last row using \texttt{tail()}
    to make sure it's meeting expectations.
  \item
    Pull out that last row using \texttt{nrow()} instead of the row
    number.
  \item
    Create a new data frame object (\texttt{trafficstops\_last}) from
    that last row.
  \end{itemize}
\item
  Use \texttt{nrow()} to extract the row that is in the middle of the
  data frame. Store the content of this row in an object named
  \texttt{trafficstops\_middle}.
\item
  Combine \texttt{nrow()} with the \texttt{-} notation above to
  reproduce the behavior of \texttt{head(trafficstops)} keeping just the
  first through 6th rows of the trafficstops dataset.
\end{enumerate}
\end{quote}

\section{Conditional subsetting}\label{conditional-subsetting-1}

Often times we need to extract a subset of a data frame based on certain
conditions. For example, if we wanted to look at traffic stops in
Tallahatchie County only we could say:

\begin{Shaded}
\begin{Highlighting}[]
\CommentTok{# the condition:}
\NormalTok{trafficstops$county_name ==}\StringTok{ "Tallahatchie County"} \CommentTok{# returns a logical vector of the length of the column}
\CommentTok{# use this vector to extract rows and all columns}
\NormalTok{trafficstops[trafficstops$county_name ==}\StringTok{ "Tallahatchie County"}\NormalTok{, ] }\CommentTok{# note the comma: we want all columns}
\CommentTok{# assign it to a new data frame}
\NormalTok{Tallahatchie_trafficstops <-}\StringTok{ }\NormalTok{trafficstops[trafficstops$county_name ==}\StringTok{ "Tallahatchie County"}\NormalTok{, ]}
\end{Highlighting}
\end{Shaded}

This is also a possibility:

\begin{Shaded}
\begin{Highlighting}[]
\NormalTok{Tallahatchie_trafficstops <-}\StringTok{ }\KeywordTok{subset}\NormalTok{(trafficstops, county_name ==}\StringTok{ "Tallahatchie County"}\NormalTok{)}
\KeywordTok{nrow}\NormalTok{(Tallahatchie_trafficstops) }\CommentTok{# 393 stops in Tallahatchie County!}
\end{Highlighting}
\end{Shaded}

\begin{verbatim}
#> [1] 393
\end{verbatim}

These commands are from the R base package. In the R Data Wrangling
workshop we will discuss a different way of subsetting using functions
from the \texttt{tidyverse} package.

\begin{quote}
Challenge

\begin{itemize}
\tightlist
\item
  Use subsetting to extract trafficstops in Hancock, Harrison, and
  Jackson Counties into a separate data frame
  \texttt{coastal\_counties}.
\item
  Using \texttt{coastal\_counties}, count the number of Black and White
  drivers in the three counties.
\item
  Bonus: How does the ratio of Black to White stops in the three coastal
  counties compare to the same ratio for stops in the entire state of
  Mississippi?
\end{itemize}
\end{quote}

\section{Adding and removing rows and
columns}\label{adding-and-removing-rows-and-columns}

To add a new column to the data frame we can use the \texttt{cbind()}
function.

\begin{Shaded}
\begin{Highlighting}[]
\NormalTok{new_col <-}\StringTok{ }\KeywordTok{row.names}\NormalTok{(trafficstops)}
\NormalTok{trafficstops_withnewcol <-}\StringTok{ }\KeywordTok{cbind}\NormalTok{(trafficstops, new_col)}
\KeywordTok{head}\NormalTok{(trafficstops_withnewcol)}
\end{Highlighting}
\end{Shaded}

\begin{verbatim}
#>              id state  stop_date       county_name county_fips
#> 1 MS-2013-00001    MS 2013-01-01      Jones County       28067
#> 2 MS-2013-00002    MS 2013-01-01 Lauderdale County       28075
#> 3 MS-2013-00003    MS 2013-01-01       Pike County       28113
#> 4 MS-2013-00004    MS 2013-01-01    Hancock County       28045
#> 5 MS-2013-00005    MS 2013-01-01     Holmes County       28051
#> 6 MS-2013-00006    MS 2013-01-01    Jackson County       28059
#>            police_department driver_gender driver_birthdate driver_race
#> 1 Mississippi Highway Patrol             M       1950-06-14       Black
#> 2 Mississippi Highway Patrol             M       1967-04-06       Black
#> 3 Mississippi Highway Patrol             M       1974-04-15       Black
#> 4 Mississippi Highway Patrol             M       1981-03-23       White
#> 5 Mississippi Highway Patrol             M       1992-08-03       White
#> 6 Mississippi Highway Patrol             F       1960-05-02       White
#>                                                 violation_raw officer_id
#> 1                     Seat belt not used properly as required       J042
#> 2                                            Careless driving       B026
#> 3 Speeding - Regulated or posted speed limit and actual speed       M009
#> 4 Speeding - Regulated or posted speed limit and actual speed       K035
#> 5 Speeding - Regulated or posted speed limit and actual speed       D028
#> 6 Speeding - Regulated or posted speed limit and actual speed       K023
#>   new_col
#> 1       1
#> 2       2
#> 3       3
#> 4       4
#> 5       5
#> 6       6
\end{verbatim}

Alternatively, we can also add a new column adding the new column name
after the \texttt{\$} sign then assigning the value, like below. Note
that this will change the original data frame, which you may not always
want to do.

\begin{Shaded}
\begin{Highlighting}[]
\NormalTok{trafficstops$row_numbers <-}\StringTok{ }\KeywordTok{c}\NormalTok{(}\DecValTok{1}\NormalTok{:}\KeywordTok{nrow}\NormalTok{(trafficstops))}
\NormalTok{trafficstops$all_false <-}\StringTok{ }\OtherTok{FALSE}  \CommentTok{# what do you think will happen here?}
\end{Highlighting}
\end{Shaded}

There is an equivalent function, \texttt{rbind()} to add a new row to a
data frame. I use this far less frequently than the column equivalent.
The one thing to keep in mind is that the row to be added to the data
frame needs to match the order and type of columns in the data frame.
Remember that R's way to store multiple different data types in one
object is a \texttt{list}. So if we wanted to add a new row to
\texttt{trafficstops} we would say:

\begin{Shaded}
\begin{Highlighting}[]
\NormalTok{new_row <-}\StringTok{ }\KeywordTok{data.frame}\NormalTok{(}\DataTypeTok{id=}\StringTok{"MS-2017-12345"}\NormalTok{, }\DataTypeTok{state=}\StringTok{"MS"}\NormalTok{, }\DataTypeTok{stop_date=}\StringTok{"2017-08-24"}\NormalTok{,}
                \DataTypeTok{county_name=}\StringTok{"Tallahatchie County"}\NormalTok{, }\DataTypeTok{county_fips=}\DecValTok{12345}\NormalTok{,}
                \DataTypeTok{police_department=}\StringTok{"MSHP"}\NormalTok{, }\DataTypeTok{driver_gender=}\StringTok{"F"}\NormalTok{, }\DataTypeTok{driver_birthdate=}\StringTok{"1999-06-14"}\NormalTok{,}
                \DataTypeTok{driver_race=}\StringTok{"Hispanic"}\NormalTok{, }\DataTypeTok{violation_raw=}\StringTok{"Speeding"}\NormalTok{, }\DataTypeTok{officer_id=}\StringTok{"ABCD"}\NormalTok{)}

\NormalTok{trafficstops_withnewrow <-}\StringTok{ }\KeywordTok{rbind}\NormalTok{(trafficstops, new_row)}
\KeywordTok{tail}\NormalTok{(trafficstops_withnewrow)}
\end{Highlighting}
\end{Shaded}

\begin{verbatim}
#>                   id state  stop_date         county_name county_fips
#> 211207 MS-2016-24293    MS 2016-07-09       George County       28039
#> 211208 MS-2016-24294    MS 2016-07-10       Copiah County       28029
#> 211209 MS-2016-24295    MS 2016-07-11      Grenada County       28043
#> 211210 MS-2016-24296    MS 2016-07-14       Copiah County       28029
#> 211211 MS-2016-24297    MS 2016-07-14       Copiah County       28029
#> 211212 MS-2017-12345    MS 2017-08-24 Tallahatchie County       12345
#>                 police_department driver_gender driver_birthdate
#> 211207 Mississippi Highway Patrol             M       1992-07-14
#> 211208 Mississippi Highway Patrol             M       1975-12-23
#> 211209 Mississippi Highway Patrol             M       1998-02-02
#> 211210 Mississippi Highway Patrol             F       1970-06-14
#> 211211 Mississippi Highway Patrol             M       1948-03-11
#> 211212                       MSHP             F       1999-06-14
#>        driver_race
#> 211207       White
#> 211208       Black
#> 211209       White
#> 211210       White
#> 211211       White
#> 211212    Hispanic
#>                                                      violation_raw
#> 211207 Speeding - Regulated or posted speed limit and actual speed
#> 211208 Speeding - Regulated or posted speed limit and actual speed
#> 211209                     Seat belt not used properly as required
#> 211210       Expired or no non-commercial driver license or permit
#> 211211                     Seat belt not used properly as required
#> 211212                                                    Speeding
#>        officer_id
#> 211207       K025
#> 211208       C033
#> 211209       D014
#> 211210       C015
#> 211211       C015
#> 211212       ABCD
\end{verbatim}

A convenient function to know about is \texttt{na.omit()}. It will
remove all rows from a data frame that have at least one column with
\texttt{NA} values.

\begin{quote}
Challenge

\begin{itemize}
\tightlist
\item
  Given the following data frame:
\end{itemize}

\begin{Shaded}
\begin{Highlighting}[]
\NormalTok{dfr <-}\StringTok{ }\KeywordTok{data.frame}\NormalTok{(}\DataTypeTok{col_1 =} \KeywordTok{c}\NormalTok{(}\DecValTok{1}\NormalTok{:}\DecValTok{3}\NormalTok{), }
                  \DataTypeTok{col_2 =} \KeywordTok{c}\NormalTok{(}\OtherTok{NA}\NormalTok{, }\OtherTok{NA}\NormalTok{, }\StringTok{"b"}\NormalTok{), }
                  \DataTypeTok{col_3 =} \KeywordTok{c}\NormalTok{(}\OtherTok{TRUE}\NormalTok{, }\OtherTok{NA}\NormalTok{, }\OtherTok{FALSE}\NormalTok{))}
\end{Highlighting}
\end{Shaded}

What would you expect the following commands to return?

\begin{Shaded}
\begin{Highlighting}[]
\KeywordTok{nrow}\NormalTok{(dfr)}
\KeywordTok{nrow}\NormalTok{(}\KeywordTok{na.omit}\NormalTok{(dfr))}
\end{Highlighting}
\end{Shaded}
\end{quote}

\section{Categorical data: factors}\label{categorical-data-factors}

When we did \texttt{str(trafficstops)} we saw that only one of the
columns are numeric (\texttt{county\_fips}), all the others are of a
special class called a \texttt{factor}. Factors are very useful and are
actually something that make R particularly well suited to working with
data, so we're going to spend a little time introducing them.

Factors are used to represent categorical data. Factors can be ordered
or unordered, and understanding them is necessary for statistical
analysis and for plotting.

Factors are stored as integers, and have labels (text) associated with
these unique integers. While factors look (and often behave) like
character vectors, they are actually integers under the hood, and you
need to be careful when treating them like strings.

Once created, factors can only contain a pre-defined set of values,
known as \emph{levels}. By default, R always sorts \emph{levels} in
alphabetical order. For instance, if you have a factor with 2 levels:

\begin{Shaded}
\begin{Highlighting}[]
\NormalTok{party <-}\StringTok{ }\KeywordTok{factor}\NormalTok{(}\KeywordTok{c}\NormalTok{(}\StringTok{"republican"}\NormalTok{, }\StringTok{"democrat"}\NormalTok{, }\StringTok{"democrat"}\NormalTok{, }\StringTok{"republican"}\NormalTok{))}
\end{Highlighting}
\end{Shaded}

R will assign \texttt{1} to the level \texttt{"democrat"} and \texttt{2}
to the level \texttt{"republican"} (because \texttt{d} comes before
\texttt{r}, even though the first element in this vector is
\texttt{"republican"}). You can check this by using the function
\texttt{levels()}, and check the number of levels using
\texttt{nlevels()}:

\begin{Shaded}
\begin{Highlighting}[]
\KeywordTok{levels}\NormalTok{(party)}
\KeywordTok{nlevels}\NormalTok{(party)}
\end{Highlighting}
\end{Shaded}

Sometimes, the order of the factors does not matter, other times you
might want to specify the order because it is meaningful (e.g., ``low'',
``medium'', ``high''), it improves your visualization, or it is required
by a particular type of analysis. Here, one way to reorder our levels in
the \texttt{party} vector would be:

\begin{Shaded}
\begin{Highlighting}[]
\NormalTok{party }\CommentTok{# current order}
\end{Highlighting}
\end{Shaded}

\begin{verbatim}
#> [1] republican democrat   democrat   republican
#> Levels: democrat republican
\end{verbatim}

\begin{Shaded}
\begin{Highlighting}[]
\NormalTok{party <-}\StringTok{ }\KeywordTok{factor}\NormalTok{(party, }\DataTypeTok{levels =} \KeywordTok{c}\NormalTok{(}\StringTok{"republican"}\NormalTok{, }\StringTok{"democrat"}\NormalTok{))}
\NormalTok{party }\CommentTok{# after re-ordering}
\end{Highlighting}
\end{Shaded}

\begin{verbatim}
#> [1] republican democrat   democrat   republican
#> Levels: republican democrat
\end{verbatim}

In R's memory, these factors are represented by integers (1, 2, 3), but
are more informative than integers because factors are self describing:
\texttt{"democrat"}, \texttt{"republican"} is more descriptive than
\texttt{1}, \texttt{2}. Which one is ``republican''? You wouldn't be
able to tell just from the integer data. Factors, on the other hand,
have this information built in. It is particularly helpful when there
are many levels (like the country names in our example dataset).

\subsection{Converting factors}\label{converting-factors}

If you need to convert a factor to a character vector, you use
\texttt{as.character(x)}.

\begin{Shaded}
\begin{Highlighting}[]
\KeywordTok{as.character}\NormalTok{(party)}
\end{Highlighting}
\end{Shaded}

Converting factors where the levels appear as numbers (such as
concentration levels, or years) to a numeric vector is a little
trickier. One method is to convert factors to characters and then
numbers. Another method is to use the \texttt{levels()} function.
Compare:

\begin{Shaded}
\begin{Highlighting}[]
\NormalTok{f <-}\StringTok{ }\KeywordTok{factor}\NormalTok{(}\KeywordTok{c}\NormalTok{(}\DecValTok{1990}\NormalTok{, }\DecValTok{1983}\NormalTok{, }\DecValTok{1977}\NormalTok{, }\DecValTok{1998}\NormalTok{, }\DecValTok{1990}\NormalTok{))}
\KeywordTok{as.numeric}\NormalTok{(f)               }\CommentTok{# wrong! and there is no warning...}
\KeywordTok{as.numeric}\NormalTok{(}\KeywordTok{as.character}\NormalTok{(f)) }\CommentTok{# works...}
\KeywordTok{as.numeric}\NormalTok{(}\KeywordTok{levels}\NormalTok{(f))[f]    }\CommentTok{# The recommended way.}
\end{Highlighting}
\end{Shaded}

Notice that in the \texttt{levels()} approach, three important steps
occur:

\begin{itemize}
\tightlist
\item
  We obtain all the factor levels using \texttt{levels(f)}
\item
  We convert these levels to numeric values using
  \texttt{as.numeric(levels(f))}
\item
  We then access these numeric values using the underlying integers of
  the vector \texttt{f} inside the square brackets
\end{itemize}

\subsection{Renaming factors}\label{renaming-factors}

When your data is stored as a factor, you can use the \texttt{plot()}
function to get a quick glance at the number of observations represented
by each factor level. Let's look at the number of blacks and whites in
the dataset:

\begin{Shaded}
\begin{Highlighting}[]
\NormalTok{## bar plot of the number of black and white drivers stopped:}
\KeywordTok{plot}\NormalTok{(trafficstops$driver_race)}
\end{Highlighting}
\end{Shaded}

\includegraphics{R-intro_files/figure-latex/driver-race-barplot-1.pdf}

There seem to be a number of individuals for which the race information
hasn't been recorded.

Additionally, for these individuals, there is no label to indicate that
the information is missing. Let's rename this label to something more
meaningful. Before doing that, we're going to pull out the data on race
and work with that data, so we're not modifying the working copy of the
data frame:

\begin{Shaded}
\begin{Highlighting}[]
\NormalTok{race <-}\StringTok{ }\NormalTok{trafficstops$driver_race}
\KeywordTok{head}\NormalTok{(race)}
\end{Highlighting}
\end{Shaded}

\begin{verbatim}
#> [1] Black Black Black White White White
#> Levels:  Black White
\end{verbatim}

\begin{Shaded}
\begin{Highlighting}[]
\KeywordTok{levels}\NormalTok{(race)}
\end{Highlighting}
\end{Shaded}

\begin{verbatim}
#> [1] ""      "Black" "White"
\end{verbatim}

\begin{Shaded}
\begin{Highlighting}[]
\KeywordTok{levels}\NormalTok{(race)[}\DecValTok{1}\NormalTok{] <-}\StringTok{ "Missing"}
\KeywordTok{levels}\NormalTok{(race)}
\end{Highlighting}
\end{Shaded}

\begin{verbatim}
#> [1] "Missing" "Black"   "White"
\end{verbatim}

\begin{Shaded}
\begin{Highlighting}[]
\KeywordTok{head}\NormalTok{(race)}
\end{Highlighting}
\end{Shaded}

\begin{verbatim}
#> [1] Black Black Black White White White
#> Levels: Missing Black White
\end{verbatim}

\begin{quote}
Challenge

\begin{itemize}
\tightlist
\item
  Rename ``Black'' to ``African American''.
\item
  Now that we have renamed the factor level to ``Missing'', can you
  recreate the barplot such that ``Missing'' is last (to the right)?
\end{itemize}
\end{quote}

---\textgreater{}

\subsection{\texorpdfstring{Using
\texttt{stringsAsFactors=FALSE}}{Using stringsAsFactors=FALSE}}\label{using-stringsasfactorsfalse}

By default, when building or importing a data frame with
\texttt{read.csv()}, the columns that contain characters (i.e., text)
are coerced (=converted) into the \texttt{factor} data type. Depending
on what you want to do with the data, you may want to keep these columns
as \texttt{character}. To do so, \texttt{read.csv()} and
\texttt{read.table()} have an argument called \texttt{stringsAsFactors}
which can be set to \texttt{FALSE}.

In most cases, it's preferable to set
\texttt{stringsAsFactors\ =\ FALSE} when importing your data, and
converting as a factor only the columns that require this data type.

Compare the output of \texttt{str(trafficstops)} when setting
\texttt{stringsAsFactors\ =\ TRUE} (default) and
\texttt{stringsAsFactors\ =\ FALSE}:

\begin{Shaded}
\begin{Highlighting}[]
\NormalTok{## Compare the difference between when the data are being read as}
\NormalTok{## `factor`, and when they are being read as `character`.}
\NormalTok{trafficstops <-}\StringTok{ }\KeywordTok{read.csv}\NormalTok{(}\StringTok{"data/MS_policing_bw.csv"}\NormalTok{, }\DataTypeTok{stringsAsFactors =} \OtherTok{TRUE}\NormalTok{)}
\KeywordTok{str}\NormalTok{(trafficstops)}
\NormalTok{trafficstops <-}\StringTok{ }\KeywordTok{read.csv}\NormalTok{(}\StringTok{"data/MS_policing_bw.csv"}\NormalTok{, }\DataTypeTok{stringsAsFactors =} \OtherTok{FALSE}\NormalTok{)}
\KeywordTok{str}\NormalTok{(trafficstops)}
\NormalTok{## Convert the column "driver_race" into a factor}
\NormalTok{trafficstops$driver_race <-}\StringTok{ }\KeywordTok{factor}\NormalTok{(trafficstops$driver_race)}
\end{Highlighting}
\end{Shaded}

\begin{quote}
Challenge

Can you predict the class for each of the columns in the following
example? Check your guesses using \texttt{str(country\_climate)}: * Are
they what you expected? Why? Why not? * What would have been different
if we had added \texttt{stringsAsFactors\ =\ FALSE} to this call? * What
would you need to change to ensure that each column had the accurate
data type?

\begin{verbatim}
```r
country_climate <- data.frame(
       country=c("Canada", "Panama", "South Africa", "Australia"),
       climate=c("cold", "hot", "temperate", "hot/temperate"),
       temperature=c(10, 30, 18, "15"),
       northern_hemisphere=c(TRUE, TRUE, FALSE, "FALSE"),
       has_kangaroo=c(FALSE, FALSE, FALSE, 1)
       )
```
\end{verbatim}
\end{quote}

The automatic conversion of data type is sometimes a blessing, sometimes
an annoyance. Be aware that it exists, learn the rules, and double check
that data you import in R are of the correct type within your data
frame. If not, use it to your advantage to detect mistakes that might
have been introduced during data entry (a letter in a column that should
only contain numbers for instance).

\section{Dates}\label{dates}

One of the most common issues that new (and experienced!) R users have
is converting date and time information into a variable that is
appropriate and usable during analyses. If you have control over your
data the best practice for dealing with date data is to ensure that each
component of your date is stored as a separate variable, i.e a separate
column for day, month, and year. However, often we do not have control
and the date is stored in one single column and with varying order and
separating characters between its components.

Using \texttt{str()}, we can see that both dates in our data frame
\texttt{stop\_date} and \texttt{driver\_birthdate} are each stored in
one column.

\begin{Shaded}
\begin{Highlighting}[]
\KeywordTok{str}\NormalTok{(trafficstops)}
\end{Highlighting}
\end{Shaded}

As an example for how to work with dates let us see if there are
seasonal differences in the number of traffic stops.

We're going to be using the \texttt{ymd()} function from the package
\textbf{\texttt{lubridate}}. This function is designed to take a vector
representing year, month, and day and convert that information to a
POSIXct vector. POSIXct is a class of data recognized by R as being a
date or date and time. The argument that the function requires is
relatively flexible, but, as a best practice, is a character vector
formatted as ``YYYY-MM-DD''.

Start by loading the required package:

\begin{Shaded}
\begin{Highlighting}[]
\KeywordTok{library}\NormalTok{(lubridate)}
\end{Highlighting}
\end{Shaded}

\begin{Shaded}
\begin{Highlighting}[]
\NormalTok{stop_date <-}\StringTok{ }\KeywordTok{ymd}\NormalTok{(trafficstops$stop_date)}
\KeywordTok{str}\NormalTok{(stop_date) }\CommentTok{# notice the 'date' class}
\end{Highlighting}
\end{Shaded}

The \texttt{ymd} function also has nicely taken care of the fact that
the original format of the date column is a factor!

We can now easily extract year, month, and date using the respective
functions: \texttt{year()}, \texttt{month()}, and \texttt{day()} like
so:

\begin{Shaded}
\begin{Highlighting}[]
\KeywordTok{plot}\NormalTok{(}\KeywordTok{factor}\NormalTok{(}\KeywordTok{year}\NormalTok{(stop_date))) }\CommentTok{#convert year to factor to plot}
\end{Highlighting}
\end{Shaded}

\includegraphics{R-intro_files/figure-latex/yearly-stops-1.pdf}

\begin{quote}
Challenge

\begin{itemize}
\tightlist
\item
  Are there more stops in certain months of the year or certain days of
  the month?
\end{itemize}
\end{quote}

\begin{Shaded}
\begin{Highlighting}[]
\KeywordTok{plot}\NormalTok{(}\KeywordTok{factor}\NormalTok{(}\KeywordTok{day}\NormalTok{(stop_date)))}
\end{Highlighting}
\end{Shaded}

\includegraphics{R-intro_files/figure-latex/plot-seasonal-stops-2.pdf}
---\textgreater{}

\begin{quote}
Challenge

\begin{itemize}
\tightlist
\item
  Determine the age of the driver in years (approximate) at the time of
  the stop:
\item
  Extract \texttt{driver\_birthdate} into a vector \texttt{birth\_date}
\item
  Create a new vector \texttt{age} with the driver's age at the time of
  the stop in years
\item
  Coerce \texttt{age} to a factor and use the \texttt{plot} function to
  check your results. What do you find?
\end{itemize}
\end{quote}

\begin{Shaded}
\begin{Highlighting}[]
\CommentTok{#or}
\KeywordTok{hist}\NormalTok{(age)}
\end{Highlighting}
\end{Shaded}

\includegraphics{R-intro_files/figure-latex/calculate-age-answer-2.pdf}
---\textgreater{}

\bibliography{packages.bib,book.bib}


\end{document}
